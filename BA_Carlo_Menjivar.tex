%  Bachelorarbeit                                           %%
%% TH Köln -Campus Gummersbach, Fak. 10)                    %%
%% 2021                                                     %%
\documentclass[a4paper,12pt,oneside]{article}
% Optionen:
% - a4paper => DIN A4-Format
% - 12pt    => Schriftgröße (weitere  
%              grundlegende Fontgrößen: 10pt, 11pt)
% - oneside => Einseitiger Druck

%% Verwendete Pakete:
\usepackage[ngerman]{babel} % für die deutsche Sprache
\usepackage{caption} % Für schönere Bildunterschriften
\usepackage[T1]{fontenc} % Schriftkodierung (Für Sonderzeichen u.a.)
\usepackage[utf8]{inputenc} % Für die direkte Eingabe von Umlauten im Editor u.a.
\usepackage{fancyhdr} % Für Kopf- und Fußzeilen
\usepackage{lscape} % Für Querformat

%% Show code in a better way
\usepackage{listings}
\usepackage{color}

\definecolor{dkgreen}{rgb}{0,0.6,0}
\definecolor{gray}{rgb}{0.5,0.5,0.5}
\definecolor{mauve}{rgb}{0.58,0,0.82}

\lstset{frame=tb,
  language=Java,
  aboveskip=3mm,
  belowskip=3mm,
  showstringspaces=false,
  columns=flexible,
  basicstyle={\small\ttfamily},
  numbers=none,
  numberstyle=\tiny\color{gray},
  keywordstyle=\color{blue},
  commentstyle=\color{dkgreen},
  stringstyle=\color{mauve},
  breaklines=true,
  breakatwhitespace=true,
  tabsize=3
}
%% Schriften (Beispiele)
%% Weitere LaTeX-Schriften im "LaTeX Font Catalogue"
%% unter: http://www.tug.dk/FontCatalogue/.
%% ACHTUNG: Ggf. müssen Schriften noch installiert 
%% werden!

% Serifen-Schriften:
\usepackage{lmodern} % Schriftart "Latin Modern"
%\usepackage{garamond} % Schriftart "Garamond"

%Sans Serif-Schriften:
%\usepackage[scaled]{uarial}
%\usepackage[scaled]{helvet}
%%--------------
\usepackage[normalem]{ulem} % Für das Unterstreichen von Text z.B. mit \uline{}
\usepackage[left=3cm,right=2cm,top=1.5cm,bottom=1cm,
textheight=245mm,textwidth=160mm,includeheadfoot,headsep=1cm,
footskip=1cm,headheight=14.599pt]{geometry} % Einrichtung der Seite 

\usepackage{graphicx} % Zum Laden von Graphiken
\graphicspath{ {./sources/} }

% INFO: Graphiken einbinden
%
% \includegraphics[scale=1.00]{dateiname}
%
% => Ausgabeformat: PDF-Dokument:
%    Es können die folgenden (Graphik-)formate eingebunden
%    werden: .jpg, .png, .pdf, .mps
% 
% => Ausgabeformat: DVI/PS:
%    Folgende (Graphik-)formate werden unterstützt:
%    .eps, .ps, .bmp, .pict, .pntg
\usepackage{epstopdf}

% Pakete für Tabellen
\usepackage{tabularx} % Einfache Tabellen
\usepackage{longtable} % Tabellen als Gleitobjekte (für die Aufteilung bei langen 
 %Tabellen über mehrere Seiten)
\usepackage{multirow} % Für das Verbinden von Zeilen innerhalb einer Tabelle mit
 % \multirow{anzahl}{*}{Text}

% (Zusatz-)Pakete für Formeln
\usepackage{amsmath}
\usepackage{amsthm}
\usepackage{amsfonts}

\usepackage{setspace} % Paket zum Setzen des Zeilenabstandes
% INFO: Zeilenabstand setzen:
%
% Befehle:
% - \singlespacing  => 1-zeilig (Standard)
% - \onehalfspacing => 1,5-zeilig
% - \doublespacing  => 2-zeilig 
\onehalfspacing % Zeilenabstand auf 1,5-zeilig setzen

% Farbboxen (für die Merkkästen in dieser Vorlage):
\usepackage{tcolorbox}
\tcbset{colback=white,colframe=orange,
        fonttitle=\bfseries}

\usepackage[colorlinks,pdfpagelabels,pdfstartview=FitH,
bookmarksopen=true,bookmarksnumbered=true,linkcolor=black,
plainpages=false,hypertexnames=false,citecolor=black]{hyperref} % Für Verlinkungen
% INFO: Verlinkungen mit dem hyperref-Paket:
%
% Die Angabe von URLs mit dem Befehl \url{} erlaubt einen
% gesonderten Umgang mit Weblinks. Denn die Links werden verlinkt.
% Auch erfolgt automatisch am Zeilenende ein Umbruch des Links.
% Es ist auch nicht erforderlich, Sonderzeichen in der URL manuell zu 
% entschärfen.
%
% TIPP: Sollte ein Umbuch bei einem Link nicht automatisch erfolgen, so kann
% das daran liegen, dass ein/mehrere Zeichen zusätzlich angegeben werden müssen,
% an dem der Link umbrochen werden kann.
% Dies kann mit folgendem Befehl erfolgen (Beispiel):
% \renewcommand*\UrlBreaks{\do-\do_}

% Das Paket "biblatex" für autom. 
% Literaturverzeichnisse:
%\usepackage{csquotes} % Für sprachangepasste Anführungszeichen
%\usepackage[backend=bibtex,style=alphabetic]  
%           {biblatex}
%\addbibresource{bib/literatur.bib}           

%%%%%%%%%%%%%%%%%%%%%%%%%%%%%%%%%%%%%%%%%%%%%
%% DOKUMENT                                %%
%%%%%%%%%%%%%%%%%%%%%%%%%%%%%%%%%%%%%%%%%%%%%

\begin{document}
% Unbeschriftetes Vorblatt (Leere Seite)
\pagestyle{empty} % Seite ohne Kopf- und Fußzeilen
\newpage % Neue Seite
\input{leereSeite} % Ausgelagerte LaTeX-Datei (hier: leereSeite.tex) einbinden

\newpage

% Deckblatt
\pagestyle{empty}
\begin{titlepage}
  \includegraphics[scale=0.20]{sources/TH_Koeln_Logo}\\
  \begin{center}
    \Large
    Technische Hochschule Köln\\
    Fakultät für Informatik und Ingenieurwissenschaften\\
    \hrule\par\rule{0pt}{2cm} % Horizontale Trennlinie  mit 2 cm Abtand nach unten erzeugen
    \LARGE
    \textsc{BACHELORARBEIT}\\
    \vspace{1cm} % Vertikaler Abstand von 1cm erzeugen
    \huge
    Kostenoptimierung für Cloud-Diensten \\
    \Large
    Ein wirtschaftlicher Ansatz für Amazon Web Services\\
    \vspace{1cm}
    \large
    Vorgelegt an der TH Köln Campus Gummersbach\\
    im Studiengang Wirtshaftsinformatik\\
    \vspace{1.0cm}
    ausgearbeitet von:\\
    \textsc{Carlo Menjivar} 11117929\\
    \vspace{1cm}
    \begin{tabular}{ll} % Einfache Tabelle ohne Rahmen, mit 2 Spalten erzeugen
      \textbf{Erster Prüfer:}  & Prof. Dr. Roman Majewski \\
      \textbf{Zweiter Prüfer:} & <Name des 2. Prüfers>    \\
    \end{tabular}
    \vspace{1cm}
    \\Gummersbach, im F<Monat der Abgabe>
  \end{center}
\end{titlepage}

\newpage

% Abstract (ACHTUNG: Abweichung zur Reihenfolge im Merkblatt!)
\begin{abstract}
  Platz für das deutsche Abstract...
\end{abstract}

\renewcommand{\abstractname}{Abstract}
\begin{abstract}
  Platz für das englische Abstract...
\end{abstract}
%<MERKKASTEN> (für die eigene Verwendung bitte entfernen
\vspace{1cm}
\begin{tcolorbox}[title={Das Abstract}]
  Bei einem Abstract handelt es sich um eine Art \textit{Zusammenfassung} Ihrer Arbeit. Diese kann in deutscher und/oder englischer Sprache verfasst werden. Mithilfe des Abstracts kann der Leser sich zügig orientieren, in wie fern Ihre Arbeit für ihn Relevanz besitzt.\\                                                                      Sprechen Sie unbedingt mit Ihrer Betreuerin/Ihrem Betreuer, ob Sie für Ihre Arbeit ein Abstract benötigen.\\
  Ein Abstract beinhaltet folgende Aspekte \footnote{ Vgl. \cite{SW11}, S. 249}:
  \begin{itemize}
    \item Ziel der Arbeit
    \item Fragestellung der Arbeit
    \item Herangezogener, theoretischer Ansatz ("Quellen")
    \item \textit{Optional:} Methodik
  \end{itemize}
\end{tcolorbox}
%</MERKKASTEN>

%<MERKKASTEN> (für die eigene Verwendung bitte entfernen
\vspace{1cm}
\begin{tcolorbox}[title={Hinweise zu dieser Dokumentvorlage}]
  \begin{itemize}
    \item Es handelt sich hierbei um eine Beispiel-Vorlage für wissenschaftliche Ausarbeitungen.
          Über die konkrete, formale Ausgestaltung Ihrer wissenschaftlichen Arbeit sprechen Sie unbedingt mit Ihre/m Betreuer/in.
    \item Unabhängig, ob Sie beispielsweise eine Bachelor-, Master- oder Hausarbeit schreiben müssen. Diese Vorlage kann als eine gute Basis für Ihre Arbeit dienen. Passen Sie einfach die Vorlage Ihren Anforderungen entsprechend an.
  \end{itemize}
\end{tcolorbox}
%</MERKKASTEN>  

\newpage

% Inhaltsverzeichnis
\tableofcontents

\newpage
\pagestyle{fancy} % Kopf- und Fußzeilen aktivieren (=> Paket "fancyhdr")

% Abbildungsverzeichnis  
% INFO: Abbildung einbinden (Beispiel):
%  \begin{figure}[h!]
%    \centering
%    \includegraphics[scale=1.00]{Pfad zum Bild}\\
%    \caption{Bildunterschrift} 
%    \label{Marke zum Referenzieren auf die Abbildung}
%  \end{figure}
\section*{Abbildungsverzeichnis}
\addcontentsline{toc}{section}{Abbildungsverzeichnis} % Manuellen Eintrag im Inhaltsverzeichnis erzeugen
\renewcommand{\listfigurename}{} % Name des Abbildungsverzeichnisses ändern
\thispagestyle{empty}
\listoffigures

\newpage
% INFO: Querverweise auf Gliederungselemente, Abbildungen 
%       & Tabellen setzen:
%
% Voraussetzung: Gesetzte Referenzmarke mit dem Befehl: \label{marke}
% 
% Referenzierung erfolgt dann mittels dem Befehl:
% \ref{marke}

\section{Einleitung}\label{kap_einleitung}
%Motivation und Ziele als subsecion?
%WEITERE QUELLE Für mOTIVATION
%https://www.gartner.com/smarterwithgartner/4-trends-impacting-cloud-adoption-in-2020
%\subsection{Einführung in das Thema (Motivation, zentrale Begriffe etc.)}
\subsection{Motivation}
%WARUM HABE ICH MICH FÜR AWS ENTSCHIEDEN?
Amazon Web Services, kurz AWS, wurde unter anderem für diese Arbeit ausgewählt wegen seiner frühen Präsenz (2006) als Cloudanbieter und seines großen Angebotes an Dienstleistungen, welche für zahlreiche Anwendungsfälle geeignet sind.
\\
Eine Recherche von Gartner positioniert AWS als Marktführer in der Magic Quadrant für Cloud-Infrastruktur und Plattform-Services 2021.
%\footnote
{\cite{G01}}
\begin{comment} GELÖSCHT, WEIL DIESE EINE BEHAUPTUNG IST (25.10.2021)
\\\\
Für viele Unternehmen ist eine große Herausforderung, die Kosten von Cloud-Diensten übersichtlich zu halten und Optimierungsmöglichkeit leicht zu erkennen. Zusätzlich besteht die Gefahr, unangenehme Überraschungen in einer Rechnung zu bekommen, weil keine Grenze für den Konsum von Cloud-Diensten festgelegt wurde. 
\end{comment}

Die Kostenoptimierung für Cloud-Dienste ist so wichtig, dass wenn keine Optimierungsmaßnahmen ergriffen werden, wird es sicherlich mehr bezahlt als bei On-Premise Systeme.
\\
\begin{quote}
    ”Indeed, if you run the cloud the same way you run your on-premise data center, you are almost certain to incur higher expenses. It is necessary to use the following key cloud cost optimization techniques in order to successfully save money on the cloud.”
{\cite{CCB}}
\end{quote}

\subsection{Problemstellung}

%Wenn ein Hotel die Vorteile von dem Cloud-Computing hätte, dann könnte dieses folgendermaßen funktionieren:
\begin{comment}
\\\\
”Heute hatten wir 17 Gäste für unsere derzeit 20 Zimmer. Für die kommende Messe am Wochenende sind wir bereit 500 Gäste zu empfangen. Nach der Messe werden wir mit unseren üblichen 20 Zimmern wie immer gut arbeiten können.”
Normalerweise bräuchte man eine große Investition zu machen, um solche kurzfristige Nachfrage zu erfüllen. Vergleichbar ist es bei traditionellen IT-Infrastrukturen, mehr Kapazitätsbedarf, würde die Anschaffung von einer neuen Hardware bedeuten.
\\\\
\end{comment}
Die Verwendung von Cloud-Diensten bringt viele Vorteile mit sich. Zum Beispiel kurzfristige Erhöhung oder Verringerung der Speicher- und Rechenkapazität, sowie Zugriff auf unterschiedliche Speicherarten, die genau an individuellen Anwendungsfälle angepasst sind. Alle diese Lösungen sind in wenigen Minuten einsatzfertig. 
\\\\
%Viele Unternehmen befürchten jedoch, dass der Wechsel von On-Premise zu On-Demand zu hohen Kosten führen könnte.
In einer Umfrage haben circa 50\% der Unternehmen die Verwaltung der Kosten für den Betrieb von Cloud-Workloads als großes Hindernis genannt. Mehr als die Hälfte der Befrachter haben geäußert, Schwierigkeiten zu haben, alle Kosten für Cloud-Workloads zu erklären. 
\begin{quote}
    „In its Stratecast Predictions 2018, Frost \& Sullivan noted that 53\% of IT leaders surveyed cited “managing costs to run cloud workloads” as a huge obstacle, and over 50\% have difficulty justifying the expenses of some public cloud workloads.“  
    {\cite{SP1}}
    \end{quote}

\begin{flushleft}
Diese Bachelorarbeit beschäftigt sich mit ebendieser Problematik, um herauszufinden, wie Unternehmen mit den passenden Werkzeugen, die Kosten ihrer Cloud-Dienste überwachen und im Blick halten können. 
Zum Beispiel wie mit frühzeitigen Benachrichtigungen alarmiert wird, wenn Systeme mehr Kosten verursachen als geplant.
\\
Außerdem sollte untersucht werden, wie sie mit der richtigen Auswahl an Diensten ihre Kosten optimieren können.
\end{flushleft}

\subsection{Einschränkungen}
Nach Angaben von Amazon Web Services ist es möglich bis zu 90\% für EC2 zu sparen, wenn EC2 Spot-Instanzen benutzt werden. Eine Preisreduzierung für Speichereinheiten ist möglich, wenn die richtige Speicherart ausgewählt wird. 
{\cite{AMZ08,AMZ09}} 
% und x\% für DB 
\\
Diese Arbeit legt den Fokus auf die Optimierung der oben genannten Diensten.
Als Überwachungswerkzeuge für die Kosten werden die AWS CloudWatch, der AWS Cost-Explorer und der AWS Trusted Advisor untersucht. 

\subsection{Fragestellung}
\begin{flushleft}
In dieser Arbeit wird versucht, die folgenden Fragen beantworten. 
\end{flushleft}

%Gestaltung Seite 28
%Buch: Die Gestaltung wiss. Arbeiten
\begin{itemize}
    \item
        Wie können Kosten bei Cloud-Diensten überwacht werden und wie lassen sie sich optimieren? 
        Am Beispiel von S3 Speichereinheiten und EC2-Instanzen.
    \item
        Welche Maßnahmen sind nötig, um unerwartet hohe Kosten bei Cloud-Diensten zu vermeiden.
       %Automatisierung, Benachrichtigungen und alles damit man der nidriegte Preis bezahlt
    \item 
        Was kann automatisiert werden, um Kosten zu vermeiden, die den Nutzern von Cloud-Diensten verursachen.  
        % Welche Grenzen können für das Budget von Cloud-Diensten festgelegt werden?
\end{itemize}
%“Forschungsfrage”, “These”, “Hypothese” oder “Annahme”, alle Synonimen
Meine Hypothese ist, dass Kosten von Cloud-Diensten unter Kontrolle gehalten und
reduziert werden können, wenn Überwachung- und Optimierungswerkzeuge eingesetzt werden.
\\\\

\subsection{Zielsetzung}
\textbf{Daraus ergeben sich für die Arbeit die folgenden Ziele:}\\ 
\begin{itemize}
    \item
        Als Erstes wird gezeigt, wie mithilfe von bestehenden Werkzeugen  die Kosten von Cloud-Diensten überwacht werden können.
    \item
        Als Nächstes wird anhand von Empfehlungen von Cloud-Experten identifiziert, welche Optimierungsmöglichkeiten bestehen.\\
\end{itemize}

%Die vorgestellten Werkzeuge werden auf eine Testumgebung eingesetzt und deren Auswirkungen im Bezug auf die Kosten %bewertet.\\

\subsection{Struktur der Arbeit}
%Schlüsselbegriffe
%Zunächst wird es eine kurze Einführung in die relevanten Begriffe geben, die für die zu untersuchenden AWS-Cloud-Services wichtig sind.

Diese Bachelorarbeit ist in folgenden Kapiteln unterteilt:\\\\
%2 Die gängigsten Cloud-Dienste, bei deren Geld verschwendet wird.
\textbf{In Kapitel 3} 
befasst sich mit dem Begriff Cloud-Economy und erläutert das Nutzen der Cloud im wirtschaftlichen Sinne. Diese dienen als Grundlage für diese Arbeit. \\\\
%4 Überwachung von Kosten
\textbf{Kapitel 4} 
zeigt die verschiedenen Zahlungsmodelle für Amazon Web Services. 
\\
Es werden Kriterien vorgestellt, die helfen, sich für das richtige Zahlungsmodell für verschiedene Szenarien zu entscheiden. 
\\\\
%5 AWS Cost Explorer und AWS-Kosten- und Nutzungsbericht
\textbf{In Kapitel 5} werden die Werkzeuge eingeführt, die zur Überwachung der Kosten von Cloud-Diensten eingesetzt werden können.
\\\\
%4 Methoden zur Kostenbremse
%4.1 S3 Intelligent-Tiering
%4.2 Instance-Scheduler für EC2 und AWS Reserved-Instance
\textbf{In Kapitel 6} auf Optimierungsmaßnahmen, (Benachrichtigungen von relevanten Ergebnissen) und Limitierung des Konsums von Cloud-Diensten, insbesondere auf EC2-Instanzen.

%Testumgebung
%Schließlich werden anhand eines Fallbeispiels in \textbf{Kapitel 5?}, die oben genannten Werkzeuge und Techniken in einer %kostenlosen Testumgebung getestet. Um die Zuverlässigkeit der Ergebnisse zu gewährleisten, wird alles gemacht, um die %Vorher- und Nachher-Szenarien vergleichbar zu machen. Dabei werden die Anzahl der Instanzen und deren Auslastung sowie %die Daten auf den Speichereinheiten berücksichtigt.
 

\newpage

\section{Grundlagen}\label{kap_grundlagen}
In diesem Grundlagenkapitel werden Erfolgschancen für Unternehmen aufgelistet, die Cloud-Dienste in ihre Geschäftsprozesse integrieren. Mit Cloud-Diensten sind die Dienste eines beliebigen Cloud-Anbieters im Allgemeinen gemeint und nicht ausschließlich AWS-Dienste. 
Es wird ebenfalls erklärt warum Kostenoptimierung und -überwachung relevant für Unternehmen sind.
\\\\
Folgende Ergebnisse könnten durch die Einführung von Überwachungs- und Optimierungsmaßnahmen erreicht werden:
\begin{itemize}
      \item
            Die Möglichkeit, die Kosten verschiedener Projekte, die über dieselbe Infrastruktur laufen, zu trennen.
            Auf diese Weise kann zwischen Projekten, die mehr, und Projekten, die weniger Kosten verursachen unterschieden werden.%Davor Kunden und nicht Projekte
      \item
            Eine beachtliche Erhöhung der finanziellen Rentabilität im Unternehmen.%[ZITAT].
      \item
            Eine geringere Ungewissheit bei der Umsetzung von cloudbasierten Systemen.
      \item
            Mehr Kontrolle über die Gesamtkosten des Betriebs, den sogenannten \textit{TCO}.\footnote{Vgl. TCO: Total Cost if Ownership}\footnote{Vgl. Ubuntu, delivered by Canonical: A business guide to hybrid/multi-cloud, S.2.\cite{CAN01}}

\end{itemize}
%Basandose en "Vor- und Nachteile der Nutzung von Cloud-Diensten (mit mobilen Endgeräten) in Organisationen und deren Einfluss auf die Nachhaltigkeit"
% Debería aclarar los aspectos principales de mi BA
% En mi caso: 

%1-Cuales son los miedos, razones y oportunidades para las empresas en la NUBE?

%\subsection{Risiken und Oportunitäten der Cloud...}\label{subsec_UabsGrund2}
%Vor- und Nachteile / ?

\subsection{Cloud Economics}\label{subsec_UabsGrund3}
%Was bietet die Cloud den Unternehmen?
%Economics of Cloud Computing
%https://d1.awsstatic.com/whitepapers/introduction-to-aws-cloud-economics-final.pdf
%[Date last review: 25.11 Isa]
\begin{flushleft}
%\textit{On-Demand Prinzip} kann zum Beispiel die Rechenkapazität je nach Bedarf angepasst werden 
\textit{Cloud Economics} befasst sich mit den Kosten und den Vorteilen von Cloud Computing und die dahinterstehenden wirtschaftlichen Grundsätzen. Anhand des \textit{Pay-as-you-go-Modell(PAYG)} können zum Beispiel nur die Cloud-Dienste in Anspruch genommen werden, die in dem Moment für das Unternehmen verbraucht werden. Damit entfällt die Notwendigkeit hohe Investitionen in Hardware zu tätigen, wie bei On-Premise-Systemen, wo Hardware im Voraus für den künftigen Bedarf.\footnote{Vgl. Anders Lisdorf, 2021, Cloud Computing Basics: a Non.-Technical Introduction, S.23\cite{CCB}} Durch den Verzicht auf Hardware entfallen die Kosten für Reparatur und Wartung. Cloud-Anbieter übernehmen viele Verwaltungsaufgaben. Laut Larry Carvalho und Matthew Marden führe dies zu einer Abnahme der nötigen Fachkräften.\footnote{Vgl. Larry Carvalho and Matthew Marden, 2015, Quantifying the Business Value of Amazon Web Services, S.1\cite{IDC01}} So ist die die Nutzung von Cloud-Diensten in unabhängiger Weise möglich; in Selbstbedienung und mit der Freiheit Dienste ohne Einschränkungen zu gebrauchen. Das bedeutet jedoch gleichzeitig, dass die Nutzerin oder der Nutzer von Cloud-Diensten Verantwortung für die anfallenden Kosten übernehmen.
\end{flushleft}
%[Grafik der Kosten On-Premise/Demand?]

\subsubsection{Skalierbarkeit}
Skalierbarkeit bezieht sich in dieser Arbeit auf die Möglichkeit, die Kapazität von Cloud-Diensten zu skalieren. Um die Leistung der IT-Infrastruktur aufrecht zu halten, ist es zum Beispiel möglich, das Serversystem so zu konfigurieren, dass es auf wechselnde Lastanforderungen reagiert.
%bietet  ist es möglich die Rechenkapazität hoch- und runterzuskalieren.
%Vertikal und Horizontal
%
%DOPPELT?Mit Auto Scaling wird sichergestellt, dass die Rechenkapazität in Zeiträumen von hoher Nachfrage automatisch hochskaliert.[AUCH RUNTER?]
% davor:  Anzahl der Amazon Server-Instanzen ABER ANZAHL VON SERVERN != Rechenkap.
% und damit Kosten minimieren.
Auf diese Weise kann Zeit mit der Verwaltung von IT-Infrastruktur eingespart werden, welche dann genutzt werden kann, um sich auf die wesentlichen Geschäftsaktivitäten zu konzentrieren.
\footnote{Mark Wilkins, 2021, AWS Certified Solutions Architect - Associate (SAA-C02), S.29.\cite{AWS1}}
%\parencite[][Seite 29]{AWS1} 
%\footnote{Vgl. [p.~29]\cite{AWS1}[]}
%OPTION 2: 
%Auf diese Weise wird Zeit mit der Verwaltung von IT-Ressourcen gespart und es kann sich auf die wesentlichen Geschäftsaktivitäten konzentrieren.
% Davor: 
%Auf diese Weise kann weniger Zeit mit der Verwaltung von IT-Ressourcen verbracht werden und sich mehr auf wesentliche Geschäftsaktivitäten konzentriert werden\footnote{Vgl. {\cite{AWS1}}, Seite 29}.
%\\\\
Dies war der Fall bei \textit{Walgreens} im Jahre 2020 in den Vereinigte Staaten.
Sie haben unter anderem 750 virtuelle Maschinen und \textit{SAP HANA} auf \textit{Azure Instanzen} migriert.
Diesbezüglich kommentierte Dan Regalado:
\begin{quote}
      By getting out of the business of managing datacenters, WBA[Walgreens Boots Alliance] can spend less time worrying about managing IT resources and more time focusing on what it’s really good at—delivering great healthcare and retail experiences to its customers. Azure also gives WBA an opportunity to better utilize the capabilities of its SAP implementation. “One of the key reasons for moving to Azure was so that we could take advantage of the scalability that SAP HANA is capable of,” explains Regalado. “Instead of using extremely big SAP HANA Large Instances, we can start using smaller VMs[virtuelle Maschinen] and then scale out.\footnote{Microsoft, 2020, Customer Story-Walgreens Boots Alliance delivers superior customer service with SAP solutions on Azure, o.S. \cite{AZU01}}
\end{quote}
So erklärte Dan Regalado, dass \textit{Walgreens} mit dem Einsatz von kleinen Instanzen und Auto-Scaling eine Serverinfrastruktur erreicht hat, die sich dem Bedarf an Rechenkapazität anpasst. 

\subsubsection{Flexibilität}% und Agilität}
Hiermit ist die Möglichkeit Cloud-Dienste in Auftrag zu geben und kündigen zu können gemeint, wenn sie nicht mehr benötigt werden. Das unter den mit dem Cloud-Anbieter vereinbarten Bedingungen.
Für Cloud-Dienste gibt es im Allgemeinen eine Vielzahl von Optionen, von denen einige Beispiele unten aufgeführt werden:
\begin{itemize}
\item
    Verschiedene Betriebssysteme, ohne oder mit Lizenzierung.
\item
    Die meistverbreiteten Programmiersprachen, unter anderem \textit{Java, C++, Go, JavaScript und Python}.{\cite{AMZ03}}
\item
    Hosting für statische Webseiten und Webanwendungen{\cite{AMZ04}}.
\item
    Populäre relationale und nicht relationale Datenbanken{\cite{AMZ10}}.           
\item
    Vielfältige Hardware-Konfigurationen.

\end{itemize}
\begin{flushleft}
Durch die Vielzahl der verfügbaren Diensten ist es möglich, Prototypen und Experimente in kurzer Zeit durchzuführen.\footnote{Vgl. IDC, 2015, Business Value of AWS S.7\cite{IDC01}} Softwareprojekte können schnell auf den Markt gebracht werden. Je nach ihrem Erfolg ist es möglich, sinnvolle Entscheidungen zu treffen. Wenn ein Projekt, aus welchen Gründen auch immer, kurzfristig eingestellt werden muss, könnten alle damit verbundenen Kosten ausfallen. Denn im Gegensatz zu On-Premise-Infrastrukturen gibt es keine Bindung an kostspielige Hardware.
      %Wenn die Neuentwicklung nicht erfolgreich war, müssen keine weitere Kosten anfallen.
      %Da die verwendete Dienste vollständig stillgelegt werden können.
\end{flushleft}

\subsubsection{Selbstbedienung}
Mit geringem Aufwand ist es möglich, Cloud-Dienste eigenständig einzurichten. Dies hat den Vorteil, dass keine weiteren Personen wie externe Spezialisten oder die Vertriebsabteilung des Cloud-Anbieters  benötigt werden.\footnote{Vgl. Anders Lisdorf, 2021, Cloud Computing Basics: a Non.-Technical Introduction, S.28\cite{CCB}}
Andererseits besteht die Gefahr, dass hohe ungewollte Kosten entstehen, wenn jemand versehentlich oder in unverantwortlicher Weise Dienstleistungen in Anspruch nimmt.    
%[TODO: ADD USE CASE WHERE THIS HAPPEND]
%LinkedIn Learning, the woman said something like that?
%BRINGT DIESE UNTERKAP. ETWAS ZUR ARBEIT BEI?

\subsubsection{Keine Vorabkosten}
%https://aws.amazon.com/de/ec2/pricing/
Das Pay-as-you-go-Modell(PAYG) wird von einer Reihe von Cloud-Anbietern angeboten.\footnote{Vgl.  Die aktuellen Marktführer im Bereich der \textit{Cloud-Computing} weltweit sind AWS, Google, Telekom und Microsoft (Vgl. Synergy Reseach Group 2019, o.S.\cite{STA6}} Dies erfordert keine Vorauszahlungen für die Nutzung von vielen Cloud-Diensten. Wenn nur für die monatlich verbrauchten Diensten bezahlt wird, verringert sich die Anfangsinvestition in die IT-Infrastruktur oder fällt ganz weg. Dies ist besonders für kleine Unternehmen interessant, die nicht über die finanziellen Mittel verfügen, um in eine IT-Infrastruktur zu investieren. Es besteht jedoch die Möglichkeit, bestimmte Beträge für die zu konsumierende Dienste im Voraus zu bezahlen. Im Unterkapitel \ref{sssec:Vorauszahlung} wird eine Berechnung der Einsparungen durch die teilweise oder vollständige Vorauszahlung der Kosten für die Nutzung von Serverinstanzen gezeigt.  

%\subsubsection{Verfügbarkeit?} t.ly/bV1z
\subsubsection{Technische Fachkompetenz}
Bei einem Einsatz von Cloud-Diensten ist zu bedenken, dass weitere Investitionen wie technische Schulungen für das Personal erforderlich werden. \textit{TÜV Rheinland} bietet zum Beispiel Kurse zur Ausbildung von Cloud Architekten an. Die Kurse dauern drei Tage und kosten 2.136,05 € pro Teilnehmer. Maßnahmen wie die genannten Kurse wirken einem der Hauptprobleme entgegen, mit denen Unternehmen bei der Migration in die Cloud konfrontiert werden. In der von Accenture im Jahr 2020 durchgeführten Umfrage gaben 38\% der Befragten an, dass fehlende Kompetenzen im Unternehmen im Bezug auf die Cloud ein Hindernis für eine Cloud-Migration ist.\footnote{Vgl. Accenture Dienstleistungen GmbH. Hohe Erwartungen an die Cloud: Hürden
meistern, Mehrwert maximieren, S.11\cite{ACC1}}
%[KOSTEN EINER IT-INFRA = SERVER+Rack]
% QUE PORTENTAJE DE LA INVERSION REPRESENTA LA INFRAESTRUCTURA DE IT EN UNA START UP Y EN UNA CORPORACION? RAZONES USAR LA NUBE(Statista)?

\subsection{Amazon Cloud-Dienste}%Sarah 6.12
Im Folgenden liegt der Fokus auf \textit{AWS-Diensten}. Einer der am häufigsten genutzten AWS-Dienste ist \textit{Amazon Elastic Computing Instances EC2}, mit dem virtuelle Maschinen erstellt werden können.\footnote{Vgl.  Kimberly Mlitz, 2021, Cloud infrastructure services vendor market share worldwide from 4th quarter 2017 to 3rd quarter 2021, o.S.\cite{STA4}} Ein weiterer wichtiger AWS-Dienst ist \textit{Amazon Simple Storage Service (S3)}, der zum Speichern von Objekten verwendet wird.\footnote{Vgl. Objekte sind in AWS die Grundeinheit in welchen Dateien in den Amazon S3-Speichereinheiten gespeichert werden. Neben den Objekten werden Metadaten, wie das Datum der Objekterstellung und das Datum der letzten Aktualisierung gespeichert.} Amazon Elastic Computing Instances EC2 werden im Folgenden in dieser Arbeit als \textit{EC2-Instanzen} und Amazon Simple Storage Service als \textit{Amazon S3} oder \textit{Amazon S3-Speichereinheiten} bezeichnet.
\\\\
Wie Lynn Langit, eine erfahrene Cloud-Architektin, feststellte, könne bis zu 80\% der AWS-Rechnung aus Gebühren für EC2-Instanzen bestehen.\footnote{Vgl. Lynn Langit, 2021, LinkedIn Learning: AWS Controlling Cost. Minute 0:20-0:45\cite{LINK2}} Laut des \textit{AWS Solutions Architekten} Daniel Peña Silva ist Amazon S3 einer der am häufigsten genutzten AWS-Dienste.\footnote{Vgl. Daniel Peña Silva, 2021, LinkedIn: Listado de todos los Servicios de AWS.\cite{LINK1}} Deshalb fokussiert sich diese Arbeit auf die Überwachungs- und Optimierungsmaßnahmen, hauptsächlich für EC2-Instanzen und Amazon S3.  %\subsubsection*{Amazon Elastic Computing Instances EC2}
%, Sektion 2 Control Costs by Service, Video AWS service type Minute 00:30}.%\subsubsection*{Amazon Simple Storage Service S3} Warum S3 t.ly/iWn3
%S3 ist der Speicherdienst für Objekte bei AWS. 
%der Rangliste vieler Informatikwebseiten und 
%BESSER: S3 WIRD IN BÜCHER WIE t.ly/IJc1 GENANNT? THIS IS A SPANISH CITAT!
\\\\
Wie aus der \autoref{fig:moreCloudStorageThanLocal} hervorgeht ist, werden darüber hinaus seit 2020 weltweit mehr Daten in Serverfarmen als auf lokalen Geräten gespeichert.\footnote{Vgl. Statista: 2020 überholt die Cloud lokale Speichermedien.\cite{STA1}} Dies bietet Vorteile im Bezug auf die Geschwindigkeit der Arbeitsabläufe, birgt aber auch Risiken wie Datendiebstahl. Das Thema Datendiebstahl wird in dieser Arbeit nicht behandelt.%; da es den Rahmen der Untersuchung sprengen würde.
\begin{figure}[h!]
      \centering
      \includegraphics[scale=0.4]{sources/moreCloudStorageThanLocal}
      \caption[2020 überholt die Cloud lokale Speichermedien]{}\label{fig:moreCloudStorageThanLocal}
      2020 überholt die Cloud lokale Speichermedien, Statista, 2019 {\cite{STA1}}
\end{figure}
\\\\
\\\\
\\\\
Dieses grundlegende Kapitel hat einige potenzielle Vorteile der Nutzung von Cloud-Diensten für Unternehmen aufgezeigt. Darüber hinaus geht der Trend in den letzten Jahren zur Nutzung von Cloud-basierten Diensten. Das nächste Kapitel befasst sich mit den Zahlungsmodellen für EC2-Instanzen und den Überlegungen, die bei der Wahl dieser Modelle in verschiedenen Szenarien zu berücksichtigen sind.
\newpage

\begin{comment}
Advantages of Cloud Technology
As the technology has matured over the last decade, companies are moving to the
cloud to lower costs, reduce complexity, and increase flexibility. The cloud
provides scalable and powerful compute solutions, low-cost, reliable storage, and addition, cloud technologies can be used to deploy solutions quickly and cost effectively around the world and on any device.
When you decouple from the data center, you’ll be able to:
x Decrease your TCO: Eliminate many of the costs related to building and
maintaining a data center or colocation deployment. Pay for only the
resources you consume.

x Reduce complexity: Reduce the need to manage infrastructure,
investigate licensing issues, or divert resources.
x Adjust capacity on the fly: Add or reduce resources, depending on
seasonal business needs, using infrastructure that is secure, reliable, and
broadly accessible.
x Reduce time to market: Design and develop new IT projects faster.
x Deploy quickly, even worldwide: Deploy applications across multiple
geographic areas.
x Increase efficiencies: Use automation to reduce or eliminate IT
management activities that waste time and resources.
x Innovate more: Spin up a new server and try out an idea. Each project
moves through the funnel more quickly because the cloud makes it faster
(and cheaper) to deploy, test, and launch new products and services.
x Spend your resources strategically: Switch to a DevOps model to free
your IT staff from operations and maintenance that can be handled by the
cloud services provider.
x Enhance security: Spend less time conducting security reviews on
infrastructure. Mature cloud providers have teams of people who focus on
security, offering best practices to ensure you’re compliant, no matter what
your industry.
\end{comment}

%\subsection{Was ist EC2? To Review}\label{subsec_UabsGrund4}
%Man kann HW und SW auswählen


% Amazon video Cloud Eco.: https://www.youtube.com/watch?v=kUNBx1MTwxw
% short explaniation https://www.youtube.com/watch?v=RI9RTbXEjLc
%3- Hard and Soft Savings https://youtu.be/Q5wSvUVPyYY?t=316
% Suche ein Buch, mit info darüber!




\vspace{1cm}
\begin{tcolorbox}[title={Das Kapitel/der Abschnitt}]
  Hierbei handelt es sich um ein Beispiel-Kapitel. Es ist zu empfehlen, dass Sie Kapitel und auch Abschnitte immer mit einer kurzen Einleitung beginnen. In dieser beschreiben Sie kurz, was den Leser in diesem Kapitel/Abschnitt erwartet. Bei einem Kapitel mit Abschnitten nehmen Sie auch inhaltlichen Bezug auf die enthaltenen Abschnitte (inklusive Referenzierung auf die Abschnittsnummerierung).
\end{tcolorbox}
%</MERKKASTEN>  

\vspace{1cm}
\begin{tcolorbox}[title={Abbildungen, Tabellen \& Co.}]
  Bei Verwendung von Tabellen und auch Abbildungen beachten Sie bitte, dass diese immer Unter-/Überschriften enthalten (inklusive einer Nummer). Im Textfluss erklären/beschreiben Sie die Abbildung bzw. die Tabelle und nehmen Bezug über einen Verweis auf die Nummer.
\end{tcolorbox}


%Eizelne Kapitel
\newpage
\section{Cloud-Dienste, bei deren Geld verschwendet wird}\label{kap_DiensteGeldFresser}
% Wie werden die Informationen die Benutzer gezeigt und wie können sie diese manipulieren
\paragraph{}

\subsection{T1}


\textbf{Große Entwickler-Community}\\
\footnote{ Vgl. \cite{SO01}}
\newpage

\subsubsection{t1.2}
Bei der Auswahl des Frameworks wollten wir so unvoreingenommen wie möglich sein, daher listen wir einige Aspekte auf, die bei Projekten mit React zu beachten sind.
\newline

\textbf{Eine zu leichte Dokumentation}\\
Aufgrund der rasanten Entwicklung ist die Dokumentation in Bezug auf die neuesten Aktualisierungen und Änderungen oft spärlich. 
\footnote{ Vgl. \cite{R01}}

\subsection{T2}
\paragraph{}



\newpage
\section{Überwachung von Kosten}\label{kap_ueberwachungVonKosten }
\subsection{Werkzeuge}
\subsubsection{Amazon CloudWatch}
Amazon CloudWatch ermöglicht die Überwachung der Leistung von Diensten. Es ist relevant für alle die Cloud-Dienste verbrauchen, wie DevOps-Manager, Ingenieure und Entwickler. Dieses Werkzeug sammelt operative Daten für die Analyse und Entscheidungsfindung.

%(How to use Spots and on-demand)
%detect CPU Utilization, with Amazon CloudWatch

\subsubsection{Trusted advisor}
%https://aws.amazon.com/de/premiumsupport/technology/trusted-advisor/
%Kunden von AWS Basic Support und AWS Developer Support können auf grundlegende Sicherheitsprüfungen und alle Prüfungen für Servicekontingente zugreifen. Kunden von AWS Business Support und AWS Enterprise Support können auf alle Prüfungen zugreifen, einschließlich Kostenoptimierung, Sicherheit, Fehlertoleranz, Leistung und Servicekontingente. 
%was macht dieses?

%Die 5 Kategorien:
\textbf{Kostenoptimierung}
\textbf{Leistung}
\textbf{...}
%IST taging EINE STRATEGIE?
%\subsubsection{Für Speicher?}
%\subsubsection{Für VMs?}
%\subsubsection{Für DB?}

\begin{quote}
  \textbf{Nachteile}
  \begin{itemize}
    \item
          a
    \item
          b

  \end{itemize}

  \footnote{Vgl. u.a. \cite{RH1}}
\end{quote}
\begin{quote}
  xxx
\end{quote}\footnote{Vgl. u.a. \cite{AX1}}

%I could compute the cost of a query, user, transaction
% https://youtu.be/qYHR_V1lvNU?t=375
\newpage

%Temporär, danach sollten wir das Package glossaries benutzen
\section{Glossar}\label{kap_glossar}
%\newglossaryentry{kiln}
%{
 % name=kiln,
  %description={German: Brennofen (m.);\\Français: fourneau (m.)},
  %plural=kilns
%}

%Make Glossary properly...
%\acrodef{VB}{Visula Basic}

\textbf{Cloud-Computing:}\\
...
\\\\
\textbf{Cloud-Dienste:}\\
...
\\\\
\textbf{On-Demand:}\\
...
\\\\
\textbf{On-Premise:}\\
...
\\\\
\textbf{Region:}\\
Die Region ist ein völlig unabhängiges und eigenständiges geografisches Gebiet. Jede Region hat mehrere, physisch getrennte und isolierte Standorte, die als Availability Zones bekannt sind. Beispiele für Regionen sind London, Dublin, Sydney, usw \footnote{\cite{AWS1}, Seite 42}.
\\\\
\textbf{Availability Zone:}\\
Eine Verfügbarkeitszone ist einfach ein Datenzentrum oder eine Sammlung von Datenzentren. Jede Verfügbarkeitszone in einer Region verfügt über eine separate Stromversorgung, Netzwerk und Konnektivität, um die Gefahr eines gleichzeitigen Ausfalls in beiden Zonen zu verringern \footnote{\cite{AWS1}, Seite 42}.
\\\\

\textbf{Instance family:}\\
Instanzfamilien sind eine Sammlung von EC2-Instanzen, die nach dem Verhältnis von Speicher, Netzwerkleistung, CPU-Größe und Speicherwerten zueinander gruppiert sind. Zum Beispiel bietet die m4-Familie von EC2 eine ausbalancierte Kombination von Rechen-, Speicher- und Netzwerkressourcen. \footnote{\cite{AWS1}, Seite 95}.
\\\\
Instagram-Story
\\\\
Tag
\\\\
Buckets
\\\\
PAYG %https://www.nimbix.net/glossary/pay-go
\newpage

\section{Zusammenfassung und Ausblick}\label{kap_zusammfAusbl}
(To-Do:)
\\Kapitelweise Kurzdarstellung der Inhalte (inklusive Referenzierung auf die \\Kapitelnummerierung) => Nach dem Motto: \textit{Was wurde wo beschrieben?}
\\Kurzdarstellung \textit{Problem – Lösungsweg – Ergebnisse}
\\Rückkopplung auf die Einleitung: Wurde die Zielstellung der Arbeit und die \\Fragestellung zufriedenstellend beantwortet?
\\Kritische Bewertung (sofern nicht bereits im Hauptteil geschehen)
\\Offene Probleme
\\Richtung der zukünftigen/möglichen Arbeiten
\\Erläuterung, warum welche Aspekte in der Arbeit nicht erläutert 

\subsection{Umweltbezogene Aspekte}
-. LA OPTIMIZACION DE COSTOS TIENE UN IMPACTO directo en las emisiones generadas por las granjas de servidores. 
Estadisticas dicen que en Europa/Alemania se generan x toneladas de CO2 provenientes de centros de computo.
\\
\subsection{Werkzeuge und Maßnahme einsetzen}
Da es in dieser Arbeit zeitlich nicht gelungen ist, die Überwachungswerkzeuge und Optimierungsmaßnahmen zu testen, müssen sie noch in einer echten Umgebung umgesetzt werden. Es wäre dann möglich zu verifizieren, ob die hier genannten Maßnahmen zur Einsparungen führen. Wenn diese gelungen sind, sollten diese Einsparungen mit der von Amazon genannten vergliechen werden.
\\
\subsection{Bewusstsein in der gesamten Organisation}
Zusätzlich zu den bisher genannten Maßnahmen ist es wichtig, dass Verantwortliche für die Kostenerzeugung Bewusstsein entwikclen. Von dem Entwickler bis zum der IT-Manager, jeder sollte wissen, dass es so einfach ist, Cloud-Dienste mit ein paar Klicks zu beauftragen. Diese können in kurzer Zeit unglückliche/ungeplante Kosten verursachen. 
\\
\subsection{Die richtige Personen(Owneship verbreiten)}
Die technischen Maßnahmen zur Überwachung und Kostenreduzierung wurden dargelegt, aber jemand muss diese Analysen, Anpassungen und Entscheidungen durchführen. 
Deshalb ist es wichtig, bestimmte Personen zu berücksichtigen, die die Verantwortung für das Geschehen in den Cloud-Systemen übernehmen. Idealerweise Menschen, die sich für das Thema interessieren und über die notwendigen Kenntnisse verfügen, um die gesetzten Ziele zu erreichen. 
%Sie redete über DIESES t.ly/XJ24

%https://content.aws.training/video/cmcfrm/de/x2/1.0.0/jwplayer.html?endpoint=https%3a%2f%2flrs.aws.training%2fTCAPI%2f&auth=Basic%20OjUzYmEwYTZmLTk0ZmMtNDAwZi1hODBlLWQ1YzA5NmNkOWY1MA%3d%3d&actor=%7b%22objectType%22%3a%22Agent%22%2c%22name%22%3a%5b%22zEjHPzGX10miDWp26Y_cLg2%22%5d%2c%22mbox%22%3a%5b%22mailto%3alms-user-zEjHPzGX10miDWp26Y_cLg2%40amazon.com%22%5d%7d&registration=2f22bc75-44b9-4175-a976-5ba4d7fe2902&activity_id=http%3a%2f%2fid.tincanapi.com%2factivity%2ftincan-prototypes%2fgolf-example&grouping=http%3a%2f%2fid.tincanapi.com%2factivity%2ftincan-prototypes%2fgolf-example&content_token=2554ffa1-5a1e-4737-9a0f-fc3abda1083e&content_endpoint=https%3a%2f%2flrs.aws.training%2fTCAPI%2fcontent%2f&externalRegistration=CompletionThresholdPercent%7c80!InstanceId%7c0!PackageId%7ccmcfrm_de_x2_1.0.0!RegistrationTimestampTicks%7c16324112989178350!SaveCompletion%7c1!TranscriptId%7cl4fipkeHAkKh-aGlnjYdug2!UserId%7czEjHPzGX10miDWp26Y_cLg2&externalConfiguration=&width=1366&height=728&left=0&top=0
%2:25
\begin{comment}

\newpage
\vspace{1cm}
\begin{tcolorbox}[title={Inhalte der \textit{Zusammenfassung und Ausblick}}]
  Das Kapitel \textit{Zusammenfassung und Ausblick} enthält folgende formale Aspekte\footnote{Vgl. \cite{BBoJ},S. 6}:
  \begin{itemize}
    \item Kapitelweise Kurzdarstellung der Inhalte (inklusive Referenzierung auf die Kapitelnummerierung) => Nach dem Motto: \textit{Was wurde wo beschrieben?}
    \item Kurzdarstellung \textit{Problem – Lösungsweg – Ergebnisse}
    \item Rückkopplung auf die Einleitung: Wurde die Zielstellung der Arbeit und die Fragestellung zufriedenstellend beantwortet?
    \item Kritische Bewertung (sofern nicht bereits im Hauptteil geschehen)
    \item Offene Probleme
    \item Richtung der zukünftigen/möglichen Arbeiten
    \item Erläuterung, warum welche Aspekte in der Arbeit nicht erläutert wurden
  \end{itemize}
Von Buch "Gestaltung"
  Schluss (Fazit)
Den Abschluss der Arbeit bildet die Zusammenfassung der wesentlichen
Ergebnisse, die folgende drei Punkte beinhaltet:
Beantwortung der Forschungsfrage, die Sie in der Einleitung
aufgeworfen haben.
Sinnstiftung der Arbeit: Für welchen Zweck sollen die Ergebnisse
verwendet werden?
Gegebenenfalls auch persönliche Bemerkungen und Bewertungen oder
ein kurzer Ausblick.
\end{tcolorbox}

\end{comment}

%FAZIT ist (Von Schribe)
%Was ist? Hier sollten die neue Erkenntnisse der Arbeit dargestellt wrden

%TIPPS: 
%Vermeide "man" und "ich"
%Addressanten sind potenzielle Cloud-Nutzer/IT-Personal.
%Es wurde AWS ausgewählt, weil...
%Wirtschaftliche Betrachtung, weil das wichtig für Firmen ist

%STRUKTUR:
%Zusammenfassung, von was gemacht wurde?
%Beanwortung der FF 
%Mehrwer für die Praxis
%Limitationen: es konnte in dieser Arbeit nicht in der Praxis geprüft werden, ob die Maßnahmen ihre Verprechen anhalten
%Weitere Forschungen: es empfehlt sich diese Maßnahmen in echte Systeme einzusetzen

\newpage

\section{Literaturverzeichnis}\label{kap_literaturverzeichnis}
% INFO: Biblatex -Ausgabe des  
% Literaturverzeichnisses (Beispiele):   
% - \printbibliography => Ausgabe ALLER 
%   Einträge
% - \printbibliography[nottype=online]
%   => Ausgabe der Einträge, bis auf die
%      "Online"-Einträge
% - \printbibliography[type=online]     
%   => Ausgabe nur der "Online"-Einträge  
%\printbibliography


% Literaturverzeichnis
% INFO: Referenzieren auf das Literaturverzeichnis:
%
% Befehl: \cite{refmarke}
% 
% "refmarke" ist die Angabe in den geschweiften Klammern bei 
% \bibitem[]{refmarke}. 
\newpage
\thispagestyle{empty}
\section{Quellenverzeichnis}
\subsection{Literatur}
\renewcommand{\refname}{} % Literaturverzeichnis ohne Bezeichnung
% Literaturverzeichnis
\begin{thebibliography}{SW11} % 2. {...} => Hier die größte /breiteste Nummer (z.B. 99) oder Kurzbeleg angeben.
  \bibitem{SW11} Stickel-Wolf, Christine; Wolf, Joachim (2011): Wissenschaftliches Lernen und Lerntechniken. Erfolgreich studieren–-gewusst wie!. Wiesbaden: Gabler.
  % TODO cite correctly
 
\end{thebibliography}

\subsection{Internetquellen}
\begin{thebibliography}{HR08} % 2. {...} => Hier die größte/breiteste Nummer (z.B. 99) oder Kurzbeleg angeben.
  \bibitem{BBoJ}Bertelsmeier, Birgit (o. J.): Tipps zum Schrei\-b\-en ei\-n\-er Ab\-sch\-luss\-ar\-beit. Fach\-hoch\-schu\-le Köln-Campus Gummersbach, Institut für Informatik. \url{http://lwibs01.gm.fh-koeln.de/blogs/bertelsmeier/files/2008/05/abschlussarbeitsbetreuung.pdf} (29.10.2013).
  \bibitem{HR08} Halfmann, Marion; Rühmann, Hans (2008): Merkblatt zur Anfertigung von Projekt-, Bachelor-, Master- und Diplomarbeiten der Fakultät 10. Fachhochschule Köln-Campus Gummersbach.\url{http://www.f10.fh-koeln.de/imperia/md/content/pdfs/studium/tipps/anleitungda270108.pdf} (29.10.2013).
  
  \bibitem{SP1}Stern, Adam, The Truth About Cloud Pricing
  \url{https://www.forbes.com/sites/forbestechcouncil/2018/11/16/the-truth-about-cloud-pricing/?sh=1f37bba42f33}(Veröffentlicht am 16.11.2018)

   \bibitem{ACC1}Accenture Dienstleistungen GmbH. (13. 11 2020). Hohe Erwartungen an die Cloud: Hürden meistern, Mehrwert maximieren
  \url{https://www.accenture.com/de-de/insights/technology/maximize-cloud-value}(Abgerufen am 12.04.2021)

 \bibitem{AM01} AWS Instance Scheduler
  \url{https://aws.amazon.com/de/solutions/implementations/instance-scheduler/}(Abgerufen am 04.2021)

 \bibitem{AM02} S3 Intelligent-Tiering Adds Archive Access Tiers
  \url{https://aws.amazon.com/de/blogs/aws/s3-intelligent-tiering-adds-archive-access-tiers/#:~:text=What%20is%20S3%20Intelligent%2DTiering}(Abgerufen am 09. 11 2020)

% \bibitem{XX}
 % \url{}(Abgerufen am/ Veröffentlicht am)
\end{thebibliography}
\newpage

\setcounter{section}{0} % Nummerierung der Gliederungsebene "section" auf 0 setzen
\renewcommand*\thesection{\Alph{section}} % Nummerierungsart für die Gliederungsebene "section" 
% auf Großbuchstaben setzen
\section{Anhang}\label{anhang}
\subsection{ANHAND X}\label{subsec_UabsAnhang}

\subsection{Verwendete Technologien}\label{subsec_UabsAnhang}
NodeJS
\\
React
\\
Bootstrap
\\
GitHub
\\
Cypress


\newpage

% Erklärung über die selbständige Abfassung der Arbeit  
\pagestyle{empty}
\section*{Erklärung über die selbständige\\Abfassung der Arbeit} % \section*{...}: das *-Symbol erlaubt, dass dieser
% Gliederungspunkt nicht ins Inhaltsverzeichnis aufgenommen wird
\addcontentsline{toc}{section}{Erklärung über die selbständige Abfassung der Arbeit}
Ich versichere, die von mir vorgelegte Arbeit selbständig verfasst zu haben.
Alle Stellen, die wörtlich oder sinngemäß aus veröffentlichten oder nicht veröffentlichten Arbeiten anderer entnommen sind,
habe ich als entnommen kenntlich gemacht.\\
Sämtliche Quellen und Hilfsmittel, die ich für die Arbeit benutzt habe, sind
angegeben. Die Arbeit hat mit gleichem Inhalt bzw. in wesentlichen Teilen noch keiner anderen Prüfungsbehörde vorgelegen.\\\\
\begin{tabular}{cp{7cm}}
                                    &             \\
                                    &             \\ \hline
  \small (Ort, Datum, Unterschrift) & \normalsize \\
\end{tabular}

%<MERKKASTEN> (für die eigene Verwendung bitte entfernen
\vspace{1cm}
\begin{tcolorbox}[title={Hinweise zur obigen \textit{Erklärung}}]
  \begin{itemize}
    \item Bitte verwenden Sie nur die Erklärung, die Ihnen Ihr \textbf{Prüfungsservice} vorgibt. Ansonsten könnte es passieren, dass Ihre Abschlussarbeit nicht angenommen wird. Fragen Sie im Zweifelsfalle bei Ihrem Prüfungsservice nach.
    \item Sie müssen \textbf{alle abzugebende Exemplare} Ihrer Abschlussarbeit unterzeichnen. Sonst wird die Abschlussarbeit nicht akzeptiert.
    \item Ein \textbf{Verstoß} gegen die unterzeichnete \textit{Erklärung} kann u.\,a. die Aberkennung Ihres akademischen Titels zur Folge haben.
  \end{itemize}
\end{tcolorbox}
%</MERKKASTEN>   

\newpage
% Unbeschriftetes Abschlussblatt (Leere Seite)
\thispagestyle{empty}
\input{leereSeite}

\end{document}

