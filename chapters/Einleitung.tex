WORK IN PROGRESS...
Wenn ein Hotel die Vorteile von dem Cloud-Computing hätte, dann könnte dieses folgendermaßen funktionieren:
\\\\
”Heute hatten wir 17 Gäste für unsere derzeit 20 Zimmer. Für die kommende Messe am Wochenende sind wir bereit 500 Gäste zu empfangen. Nach der Messe werden wir mit unseren üblichen 20 Zimmern wie immer gut arbeiten können.”
Normalerweise bräuchte man eine große Investition zu machen, um solche kurzfristige Nachfrage zu erfüllen. Vergleichbar ist es bei traditionellen IT-Infrastrukturen, mehr Kapazitätsbedarf, würde die Anschaffung von einer neuen Hardware bedeuten.
\\\\
Die Verwendung von Cloud-Diensten bringt mit sich viele Vorteile, wie zum Beispiel kurzfristige Erhöhung oder Verringerung der Speicher- und Rechenkapazität oder Zugriff auf unterschiedliche Speicherarten, die genau an individuellen Anwendungsfälle passen. Alle diese Lösungen sind in wenigen Minuten erreichbar. Viele Unternehmen befürchten jedoch, dass der Wechsel von On-Premise zu On-Demand zu hohen Kosten führen könnte.
\\\\
Für viele Unternehmen ist eine große Herausforderung, die Kosten von Cloud-Diensten übersichtlich zu halten und Optimierungsmöglichkeit leicht zu erkennen. Zusätzlich besteht das Problem, keine feste Grenze für den Konsum von Cloud-Diensten festlegen zu können, damit man keine unangenehme Überraschung in einer Rechnung bekommt. 
\\\\
Diese Bachelorarbeit beschäftigt sich mit dieser Problematik, um herauszufinden, wie Cloud-Nutzer mit den passenden Werkzeugen, die Kosten ihrer Cloud-Dienste überwachen und im Blick halten können. Außerdem sollte untersucht werden, wie Cloud-Nutzer mit der richtigen Auswahl an Diensten ihre Kosten optimieren können.
\subsection{Einführung in das Thema (Motivation, zentrale Begriffe etc.)}
\subsection{Hinführung zu den Ergebnissen}
\subsection{Ggf. Angabe des Schwerpunktes}
\subsection{Ggf. Einschränkungen darlegen}
\subsection{Problemstellung}
\subsection{Zielstellung der Arbeit}
\subsection{Fragestellung der Arbeit}

\subsection{Struktur der Arbeit}
Dieser Projektbericht ist in folgenden Kapitel unterteilt:
\textbf{Kapitel 2}
\\\\
\textbf{Kapitel 3}
\\\\
%Benutzeroberfläche
\textbf{Kapitel 4?} 
\\\\
%Serveranfragen
\textbf{Kapitel 5?} 
\\\\
\\\\
%Qualitätssicherung
\textbf{Kapitel 6?} 
\\