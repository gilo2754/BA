%Motivation und Ziele als subsecion?
%WEITERE QUELLE Für mOTIVATION
%https://www.gartner.com/smarterwithgartner/4-trends-impacting-cloud-adoption-in-2020
%\subsection{Einführung in das Thema (Motivation, zentrale Begriffe etc.)}
\subsection{Motivation}
%WARUM HABE ICH MICH FÜR AWS ENTSCHIEDEN?
Amazon Web Services, kurz AWS, wurde unter anderem für diese Arbeit ausgewählt wegen seiner frühen Präsenz (2006) als Cloudanbieter und seines großen Angebotes an Dienstleistungen, welche für zahlreiche Anwendungsfälle geeignet sind.
\\
Eine Recherche von Gartner positioniert AWS als Marktführer in der Magic Quadrant für Cloud-Infrastruktur und Plattform-Services 2021.
%\footnote
{\cite{G01}}
\begin{comment} GELÖSCHT, WEIL DIESE EINE BEHAUPTUNG IST (25.10.2021)
\\\\
Für viele Unternehmen ist eine große Herausforderung, die Kosten von Cloud-Diensten übersichtlich zu halten und Optimierungsmöglichkeit leicht zu erkennen. Zusätzlich besteht die Gefahr, unangenehme Überraschungen in einer Rechnung zu bekommen, weil keine Grenze für den Konsum von Cloud-Diensten festgelegt wurde. 
\end{comment}

Die Kostenoptimierung für Cloud-Dienste ist so wichtig, dass wenn keine Optimierungsmaßnahmen ergriffen werden, wird es sicherlich mehr bezahlt als bei On-Premise Systeme.
\\
\begin{quote}
    ”Indeed, if you run the cloud the same way you run your on-premise data center, you are almost certain to incur higher expenses. It is necessary to use the following key cloud cost optimization techniques in order to successfully save money on the cloud.”
{\cite{CCB}}
\end{quote}

\subsection{Problemstellung}

%Wenn ein Hotel die Vorteile von dem Cloud-Computing hätte, dann könnte dieses folgendermaßen funktionieren:
\begin{comment}
\\\\
”Heute hatten wir 17 Gäste für unsere derzeit 20 Zimmer. Für die kommende Messe am Wochenende sind wir bereit 500 Gäste zu empfangen. Nach der Messe werden wir mit unseren üblichen 20 Zimmern wie immer gut arbeiten können.”
Normalerweise bräuchte man eine große Investition zu machen, um solche kurzfristige Nachfrage zu erfüllen. Vergleichbar ist es bei traditionellen IT-Infrastrukturen, mehr Kapazitätsbedarf, würde die Anschaffung von einer neuen Hardware bedeuten.
\\\\
\end{comment}
Die Verwendung von Cloud-Diensten bringt viele Vorteile mit sich. Zum Beispiel kurzfristige Erhöhung oder Verringerung der Speicher- und Rechenkapazität, sowie Zugriff auf unterschiedliche Speicherarten, die genau an individuellen Anwendungsfälle angepasst sind. Alle diese Lösungen sind in wenigen Minuten einsatzfertig. 
\\\\
%Viele Unternehmen befürchten jedoch, dass der Wechsel von On-Premise zu On-Demand zu hohen Kosten führen könnte.
In einer Umfrage haben circa 50\% der Unternehmen die Verwaltung der Kosten für den Betrieb von Cloud-Workloads als großes Hindernis genannt. Mehr als die Hälfte der Befrachter haben geäußert, Schwierigkeiten zu haben, alle Kosten für Cloud-Workloads zu erklären. 
\begin{quote}
    „In its Stratecast Predictions 2018, Frost \& Sullivan noted that 53\% of IT leaders surveyed cited “managing costs to run cloud workloads” as a huge obstacle, and over 50\% have difficulty justifying the expenses of some public cloud workloads.“  
    {\cite{SP1}}
    \end{quote}

\begin{flushleft}
Diese Bachelorarbeit beschäftigt sich mit ebendieser Problematik, um herauszufinden, wie Unternehmen mit den passenden Werkzeugen, die Kosten ihrer Cloud-Dienste überwachen und im Blick halten können. 
Zum Beispiel wie mit frühzeitigen Benachrichtigungen alarmiert wird, wenn Systeme mehr Kosten verursachen als geplant.
\\
Außerdem sollte untersucht werden, wie sie mit der richtigen Auswahl an Diensten ihre Kosten optimieren können.
\end{flushleft}

\subsection{Fragestellung}
\begin{flushleft}
In dieser Arbeit wird versucht, die folgenden Fragen beantworten. 
\end{flushleft}

%Gestaltung Seite 28
%Buch: Die Gestaltung wiss. Arbeiten
\begin{itemize}
    \item
        Wie können Kosten bei Cloud-Diensten überwacht werden und wie lassen sie sich optimieren? 
        Am Beispiel von S3 Speichereinheiten und EC2-Instanzen.
    \item
        Welche Maßnahmen sind nötig, um unerwartet hohe Kosten bei Cloud-Diensten zu vermeiden.
       %Automatisierung, Benachrichtigungen und alles damit man der nidriegte Preis bezahlt
    \item 
        Was kann automatisiert werden, um Kosten zu vermeiden, die den Nutzern von Cloud-Diensten verursachen.  
        % Welche Grenzen können für das Budget von Cloud-Diensten festgelegt werden?
\end{itemize}
%“Forschungsfrage”, “These”, “Hypothese” oder “Annahme”, alle Synonimen
Meine Hypothese ist, dass Kosten von Cloud-Diensten unter Kontrolle gehalten und
reduziert werden können, wenn Überwachung- und Optimierungswerkzeuge eingesetzt werden.
\\\\

\subsection{Zielsetzung}
\textbf{Daraus ergeben sich für die Arbeit die folgenden Ziele:}\\ 
\begin{itemize}
    \item
        Als Erstes wird gezeigt, wie mithilfe von bestehenden Werkzeugen  die Kosten von Cloud-Diensten überwacht werden können.
    \item
        Als Nächstes wird anhand von Empfehlungen von Cloud-Experten identifiziert, welche Optimierungsmöglichkeiten bestehen.\\
\end{itemize}

%Die vorgestellten Werkzeuge werden auf eine Testumgebung eingesetzt und deren Auswirkungen im Bezug auf die Kosten %bewertet.\\

\subsection{Einschränkungen}
Nach Angaben von Amazon Web Services ist es möglich bis zu 90\% für EC2 zu sparen, wenn EC2 Spot-Instanzen benutzt werden. Eine Preisreduzierung für Speichereinheiten ist möglich, wenn die richtige Speicherart ausgewählt wird. 
{\cite{AMZ08,AMZ09}} 
% und x\% für DB 
\\
Diese Arbeit legt den Fokus auf die Optimierung der oben genannten Diensten.
Als Überwachungswerkzeuge für die Kosten werden die AWS CloudWatch, der AWS Cost-Explorer und der AWS Trusted Advisor untersucht. 

\subsection{Struktur der Arbeit}
%Schlüsselbegriffe
%Zunächst wird es eine kurze Einführung in die relevanten Begriffe geben, die für die zu untersuchenden AWS-Cloud-Services wichtig sind.

Diese Bachelorarbeit ist in folgenden Kapiteln unterteilt:\\\\
%2 Die gängigsten Cloud-Dienste, bei deren Geld verschwendet wird.
\textbf{In Kapitel 3} 
befasst sich mit dem Begriff Cloud-Economy und erläutert das Nutzen der Cloud im wirtschaftlichen Sinne. Diese dienen als Grundlage für diese Arbeit. \\\\
%4 Überwachung von Kosten
\textbf{Kapitel 4} 
zeigt die verschiedenen Zahlungsmodelle für Amazon Web Services. 
\\
Es werden Kriterien vorgestellt, die helfen, sich für das richtige Zahlungsmodell für verschiedene Szenarien zu entscheiden. 
\\\\
%5 AWS Cost Explorer und AWS-Kosten- und Nutzungsbericht
\textbf{In Kapitel 5} werden die Werkzeuge eingeführt, die zur Überwachung der Kosten von Cloud-Diensten eingesetzt werden können.
\\\\
%4 Methoden zur Kostenbremse
%4.1 S3 Intelligent-Tiering
%4.2 Instance-Scheduler für EC2 und AWS Reserved-Instance
\textbf{In Kapitel 6} auf Optimierungsmaßnahmen, (Benachrichtigungen von relevanten Ergebnissen) und Limitierung des Konsums von Cloud-Diensten, insbesondere auf EC2-Instanzen.

%Testumgebung
%Schließlich werden anhand eines Fallbeispiels in \textbf{Kapitel 5?}, die oben genannten Werkzeuge und Techniken in einer %kostenlosen Testumgebung getestet. Um die Zuverlässigkeit der Ergebnisse zu gewährleisten, wird alles gemacht, um die %Vorher- und Nachher-Szenarien vergleichbar zu machen. Dabei werden die Anzahl der Instanzen und deren Auslastung sowie %die Daten auf den Speichereinheiten berücksichtigt.
 
