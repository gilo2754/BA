%Motivation und Ziele als subsecion?
%WEITERE QUELLE Für mOTIVATION
%https://www.gartner.com/smarterwithgartner/4-trends-impacting-cloud-adoption-in-2020
%\subsection{Einführung in das Thema (Motivation, zentrale Begriffe etc.)}
\subsection{Problemstellung}
Wenn ein Hotel die Vorteile von dem Cloud-Computing hätte, dann könnte dieses folgendermaßen funktionieren:
\\\\
”Heute hatten wir 17 Gäste für unsere derzeit 20 Zimmer. Für die kommende Messe am Wochenende sind wir bereit 500 Gäste zu empfangen. Nach der Messe werden wir mit unseren üblichen 20 Zimmern wie immer gut arbeiten können.”
Normalerweise bräuchte man eine große Investition zu machen, um solche kurzfristige Nachfrage zu erfüllen. Vergleichbar ist es bei traditionellen IT-Infrastrukturen, mehr Kapazitätsbedarf, würde die Anschaffung von einer neuen Hardware bedeuten.
\\\\
Die Verwendung von Cloud-Diensten bringt mit sich viele Vorteile, wie zum Beispiel kurzfristige Erhöhung oder Verringerung der Speicher- und Rechenkapazität oder Zugriff auf unterschiedliche Speicherarten, die genau an individuellen Anwendungsfälle passen. Alle diese Lösungen sind in wenigen Minuten erreichbar. Viele Unternehmen befürchten jedoch, dass der Wechsel von On-Premise zu On-Demand zu hohen Kosten führen könnte.
\\\\
Für viele Unternehmen ist eine große Herausforderung, die Kosten von Cloud-Diensten übersichtlich zu halten und Optimierungsmöglichkeit leicht zu erkennen. Zusätzlich besteht das Problem, keine feste Grenze für den Konsum von Cloud-Diensten festlegen zu können, damit man keine unangenehme Überraschung in einer Rechnung bekommt. 
\\\\
Die Optimierung der Kosten, die durch Cloud-Dienste entstehen, ist so wichtig, dass wenn keine Optimierungsmaßnahmen ergriffen werden, wird es mit Sicherheit mehr bezahlt werden als bei On-Premise Systeme.
\\
\begin{quote}
    ”Indeed, if you run the cloud the same way you run your on-premise data center, you are almost certain to incur higher expenses. It is necessary to use the following key cloud cost optimization techniques in order to successfully save money on the cloud.”
\footnote{Vgl. u.a. \cite{CCB}}
\end{quote}

Diese Bachelorarbeit beschäftigt sich mit dieser Problematik, um herauszufinden, wie Firmen mit den passenden Werkzeugen, die Kosten ihrer Cloud-Dienste überwachen und im Blick halten können. Außerdem sollte untersucht werden, wie sie mit der richtigen Auswahl an Diensten ihre Kosten optimieren können.
\newpage

\subsection{Hinführung zu den Ergebnissen. / Angabe des Schwerpunktes / Einschränkungen darlegen}
Nach Angaben von Amazon Web Services ist es möglich bis zu x\% der Kosten für S3, z\% für EC2 und x\% für DB zu sparen.
\\
Diese Arbeit liegt den Fokus auf die Optimierung der oben genannten Diensten.
Als Überwachungwerkzeuge für die Kosten werden das A und das B untersucht. 
\\
\subsection{Fragestellung und Ziel der Arbeit}
\begin{quote}
„Bei der Umfrage von Stratecast Predictions 2018, Frost \& Sullivan, nannten 53 \% der IT-Führungskräfte die Verwaltung der Kosten für den Betrieb von Cloud-Workloads als großes Hindernis.“  
\footnote{Vgl. u.a. \cite{SP1}}
\end{quote}
Ausgehend von dieser Beobachtung untersucht die Arbeit die folgenden Fragen. 
%Gestaltung Seite 28
%Buch: Die Gestaltung wiss. Arbeiten
\begin{itemize}
    \item
        Wie können Kosten bei Cloud-Diensten überwacht und wie lassen sich (unerwartete) Kosten bei Cloud-Diensten vermeiden/optimieren? am Beispiel [S3, EC2 und ein DB.]
    \item
        Welche Grenzen können für das Budget von Cloud-Diensten festgelegt werden?
\end{itemize}
%“Forschungsfrage”, “These”, “Hypothese” oder “Annahme”, alle Synonimen
Meine Hypothese ist, dass Kosten von Cloud-Diensten unter Kontrolle gehalten und
reduziert werden können, wenn Überwachung- und Optimierungswerkzeuge eingesetzt werden.
\\\\
\textbf{Daraus ergeben sich für die Arbeit die folgenden Ziele:}\\ 
Als erstes wird gezeigt, wie mit Hilfe von bestehenden Werkzeugen unerwartete Kosten von Cloud-Diensten vermieden werden können.\\\\
Als nächstes wird anhand von Empfehlungen von Cloud-Experten identifiziert, welche Optimierungsmöglichkeiten bestehen.\\
%Die vorgestellten Werkzeuge werden auf eine Testumgebung eingesetzt und deren Auswirkungen im Bezug auf die Kosten %bewertet.\\
\newpage
\subsection{Struktur der Arbeit}
%Schlüsselbegriffe
Zunächst wird es eine kurze Einführung in die relevanten Begriffe geben, die für die zu untersuchenden AWS-Cloud-Services wichtig sind.

Diese Bachelorarbeit ist in folgenden Kapitel unterteilt:\\\\
%2 Die gängigsten Cloud-Dienste, bei deren Geld verschwendet wird.
%Zahlungsmodelle
\textbf{Kapitel X?} 
zeigt die verschiedenen Zahlungs-/Preismodelle für Amazon Web Services. 
\\\\
\textbf{In Kapitel 2} werden die Cloud-Dienste vorgestellt, die im Hinblick auf das Kostenmanagement besondere Aufmerksamkeit erfordern.
\\\\
%3 Überwachung von Kosten
%3.1 AWS Cost Explorer und AWS-Kosten- und Nutzungsbericht
\textbf{In Kapitel 3} werden die Werkzeuge eingeführt, mit denen die Kosten, die Cloud-Dienste verursachen, überwacht werden können. Der AWS-Kosten- und Nutzungsbericht schlüsselt die Nutzung auf Organisationsebene nach Produktcode, Nutzungsart und Betrieb auf. Dadurch erhält man eine bessere Idee, welche Dienste am meisten Ressourcen verbrauchen. Auf diese Weise ist man in der Lage, weitere Maßnahmen zu ergreifen.
\\\\
%4 Methoden zur Kostenbremse
%4.1 S3 Intelligent-Tiering
%4.2 Instance-Scheduler für EC2 und AWS Reserved-Instance
Als nächstes wird in \textbf{Kapitel 4?} auf die verfügbaren Werkzeuge zur Optimierung und Limitierung des Konsums von EC2-Instanzen und S3 Speichereinheiten eingegangen. Der AWS Instance Scheduler ist eine Lösung von AWS, die ermöglicht, benutzerdefinierte Start- und Stoppzeitpläne für Amazon EC2- und Relational Datenbanken-Instanzen einfach zu konfigurieren. Die Verwendung von Instance-Scheduler für EC2 kann im Gegensatz zu einer 24-Stunden-Ausführung bis zu 70 \% der Kosten einsparen.
\\\\
AWS Reserved-Instance ermöglicht, die Kapazität von EC2-Instanzen zu reservieren und von einem vergünstigten Tarif zu profitieren. Mit Hilfe von S3 Intelligent-Tiering lässt sich Speicher optimieren, indem die passende Speicherart automatisch nach Kriterien wie Zugriffshäufigkeit verwendet wird.
\\\\
%Testumgebung
%Schließlich werden anhand eines Fallbeispiels in \textbf{Kapitel 5?}, die oben genannten Werkzeuge und Techniken in einer %kostenlosen Testumgebung getestet. Um die Zuverlässigkeit der Ergebnisse zu gewährleisten, wird alles gemacht, um die %Vorher- und Nachher-Szenarien vergleichbar zu machen. Dabei werden die Anzahl der Instanzen und deren Auslastung sowie %die Daten auf den Speichereinheiten berücksichtigt.
 
