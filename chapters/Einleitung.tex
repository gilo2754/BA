%Motivation und Ziele als subsecion?
%WEITERE QUELLE Für mOTIVATION
%https://www.gartner.com/smarterwithgartner/4-trends-impacting-cloud-adoption-in-2020
%\subsection*{Einführung in das Thema (Motivation, zentrale Begriffe etc.)}
\subsection{Motivation}
%\addcontentsline{toc}{subsection}{Motivation}
%WARUM HABE ICH MICH FÜR AWS ENTSCHIEDEN?
%Wie viele Firmen wechseln von On-Premise zu Cloud in DE jährlich?
[Persönliches Statement... wieso schreib ich das hier eigentlich ;)] Die zunehmende Digitalisierung von Geschäftsmodellen, die auch durch die Corona-Pandemie vorangetrieben wird, lässt Cloud-basierte Applikationen an Bedeutung gewinnen.\footnote{Es sei an diser Stelle darauf hingewiesen, dass in diesem Kontext Ahrens die Bedeutung Cloud-basierter Anwendungen im Bereich von deutschen Handelsunternehmen untersuchte (Vgl. Ahrens 2021)\cite{STA3}, sowie das \textit{ifo Institut} anschaulich die strukturellen Veränderungen von der Corona-Pandemie auf den Arbeitsalltag in Deutschland nachzeichnete (Vgl. ifo Institut 2020)\cite{STA2}.} Als direkte Folge davon ist die Nachfrage nach Server- und Speicherkapazität gestiegen.
%Dies wird auch von 48\% der befragten Handelsunternehmen im Jahr 2021 bestätigt.
Die Relevanz von \textit{Amazon Web Services}, kurz AWS, in dem Bereich der Cloud-Computing ergibt sich aus einer vor kurzem veröffentlichte Studie von Raj Bala et al.. Diese wies eindrücklich daraufhin, dass AWS der aktuell weltweit führende Cloud-Anbieter anhand ihrer Klassifikation (\textit{Magic Quadrant}\footnote{ Laut Gartner stellt der Magic Quadrant eine zweidimensionale Matrix mit vier Quadranten dar. Jeder Quadrant steht für einen Unternehmenstypus im Markt. Im Uhrzeigersinn von links unten beginnend sind dies: \textit{Nichenanbieter, Herausforderer, Marktführer }und \textit{Visionäre}}) für Cloud-Infrastruktur und Plattform-Services sei (Bala et al, 2021, o.S.,\cite{G01}).
AWS erscheint nicht nur aus diesem Grund als Fallbeispiel für diese Arbeit passend, weitere bedeutsame Faktoren sind seine frühe Präsenz (2006) als Cloudanbieter und seines großen Angebotes an Cloud-Diensten, welche für zahlreiche Anwendungsfälle geeignet sind.\footnote{ Die aktuellen Marktführer im Bereich der \textit{Cloud-Computing} weltweit sind AWS, Google, Telekom und Microsoft (Vgl. Synergy Reseach Group 2019, o.S.\cite{STA6})}
\\\\

\begin{comment} GELÖSCHT, WEIL DIESE EINE BEHAUPTUNG IST (25.10.2021)
    \\\\
    Für viele Unternehmen ist eine große Herausforderung, die Kosten von Cloud-Diensten übersichtlich zu halten und Optimierungsmöglichkeit leicht zu erkennen. Zusätzlich besteht die Gefahr, unangenehme Überraschungen in einer Rechnung zu bekommen, weil keine Grenze für den Konsum von Cloud-Diensten festgelegt wurde. 
    \end{comment}
\subsection{Problemstellung}
%\addcontentsline{toc}{subsection}{Problemstellung}
%Wenn ein Hotel die Vorteile von dem Cloud-Computing hätte, dann könnte dieses folgendermaßen funktionieren:
\begin{comment}
\\\\
”Heute hatten wir 17 Gäste für unsere derzeit 20 Zimmer. Für die kommende Messe am Wochenende sind wir bereit 500 Gäste zu empfangen. Nach der Messe werden wir mit unseren üblichen 20 Zimmern wie immer gut arbeiten können.”
Normalerweise bräuchte man eine große Investition zu machen, um solche kurzfristige Nachfrage zu erfüllen. Vergleichbar ist es bei traditionellen IT-Infrastrukturen, mehr Kapazitätsbedarf, würde die Anschaffung von einer neuen Hardware bedeuten.
\\\\
\end{comment}
%Die Verwendung von Cloud-Diensten bringt viele Vorteile mit sich. Zum Beispiel kurzfristige Erhöhung oder Verringerung der Speicher- und Rechenkapazität, sowie Zugriff auf unterschiedliche Speicherarten, die genau an individuelle Anwendungsfälle angepasst sind. All diese Lösungen sind in wenigen Minuten einsatzfertig. 
%\\\\
%Viele Unternehmen befürchten jedoch, dass der Wechsel von On-Premise zu On-Demand zu hohen Kosten führen könnte.
%In einer Umfrage haben circa 50\% der Unternehmen die Verwaltung der Kosten für den Betrieb von Cloud-Workloads als großes Hindernis genannt. Mehr als die Hälfte der Befragten haben geäußert, dass sie Schwierigkeiten haben, alle Kosten für Cloud-Workloads zu rechtfertigen.
Adam Stern wies in dem \textit{Forbes}-Magazin daraufhin, dass ungefähr die Hälfte der US-amerikanischen Unternehmen Schwierigkeiten hätten ihre Kosten zu begründen (Stern 2018, o.S.). 
\begin{quote}
    „In its Stratecast Predictions 2018, Frost \& Sullivan noted that 53\% of IT leaders surveyed cited “managing costs to run cloud workloads” as a huge obstacle, and over 50\% have difficulty justifying the expenses of some public cloud workloads.“  
    \footnote{Stern, Adam, The Truth About Cloud Pricing.\cite{SP1}}
\end{quote}
Darüber hinaus weist Tobias Regenfuß und Jochen Malinowski
von Accenture GmbH in einer Untersuchung, dass es den Unternehmen an fachlichem Know-How in Cloud-Computing mangelte. Diese stelle eine der größten Hindernisse dar, um einen Wechsel von On-Premise- zu Cloud-basierten Systemen gewährleisten zu können\footnote{Regenfuß und MalinowskiStern, Hohe Erwartungen an die Cloud: Hürden meistern, Mehrwert maximieren. 2020 o.S.(Webversion) oder S.11 in der PDF-Version auf Englisch\cite{ACC1}}.
\\\\
Kostenoptimierung für Cloud-Dienste ist ein wichtiger Punkt, da man ohne Optimierungsmaßnahmen mit höheren Kosten rechnen müsse als bei On-Premise Systemen(Anders Lisdorf\footnote{Anders Lisdorf. Cloud Computing Basics: a Non.-Technical Introduction. S.152. \cite{CCB}}).
\\
([Rev] SOLLTE DIE UNTERE DIREKTE ZITAT WEG)
\begin{quote}
    ”Indeed, if you run the cloud the same way you run your on-premise data center, you are almost certain to incur higher expenses. It is necessary to use the following key cloud cost optimization techniques in order to successfully save money on the cloud.”
    \footnote{Anders Lisdorf. Cloud Computing Basics: a Non.-Technical Introduction. S.152. \cite{CCB}}
\end{quote}
\begin{flushleft}
Diese Bachelorarbeit beschäftigt sich mit ebendieser Problematik, um herauszufinden, wie Unternehmen mit den passenden Werkzeugen die Kosten ihrer Cloud-Dienste überwachen und im Blick behalten können. %Zum Beispiel können frühzeitige Benachrichtigungen alarmieren, wenn Cloud-Dienste mehr Kosten verursachen als geplant.
\\
Außerdem sollte untersucht werden, wie mit der richtigen Auswahl an Diensten Kosten optimiert werden. 
%In dieser Arbeit wird versucht zu beantworten, wie Kosten bei Cloud-Diensten überwacht werden können. Auf Grundlage dieser Information werden Optimierungsmöglichkeiten untersucht. 
Es wird untersucht, welche Maßnahmen nötig sind, um unerwartet hohe Kosten bei Cloud-Diensten zu vermeiden. Darüber hinaus werden Empfehlungen von Cloud-Experten berücksichtigt, um Kosten von Cloud-Diensten zu minimieren. Diese Arbeit untersucht speziell die Kostenoptimierung  von S3-Speichereinheiten und EC2-Server-Instanzen mithilfe von folgenden Überwachungswerkzeuge: Cost-Explorer, CloudWatch und Trusted Advisor.
\end{flushleft}


\subsection{Zielsetzung}
%\addcontentsline{toc}{subsection}{Zielsetzung}
Die vorliegende Arbeit betrachtet die von AWS angebotenen Überwachungswerkzeuge, um ein tiefergehendes Verständnis der Entstehung von Kosten durch die Nutzung von Cloud-Diensten zu gewährleisten. Mit den von AWS zur Verfügung gestellten Maßnahmen sollen die Nutzung und damit die Kosten von Cloud-Diensten reduziert werden.
%\textbf{Daraus ergeben sich für die Arbeit die folgenden Ziele:}%\\ 
%\%begin{itemize}
 %   \item
 %       Als Erstes wird gezeigt, wie mithilfe von bestehenden %Werkzeugen  die Kosten von Cloud-Diensten überwacht %werden können.
  %  \item
 %       Als Nächstes wird anhand von Empfehlungen von Cloud-Experten identifiziert, welche Optimierungsmöglichkeiten bestehen.\\
%\end{itemize}

%Die vorgestellten Werkzeuge werden auf eine Testumgebung eingesetzt und deren Auswirkungen im Bezug auf die Kosten %bewertet.\\
\begin{comment}
\subsection*{Einschränkungen}
\addcontentsline{toc}{subsection}{Einschränkungen}

Der Schwerpunkt dieser Arbeit liegt auf EC2-Instanzen, da diese in der Regel den größten Anteil an der Rechnung ausmachen.
An zweiter Stelle stehen S3-Speichereinheiten, weil sie einen erheblichen Teil der Kosten darstellen.
%STATISTEN DIE DAS BELEGEN?

%Nach Angaben von Amazon Web Services ist es möglich bis zu 90% für EC2 zu sparen, wenn EC2 Spot-Instanzen benutzt werden. 
%Eine Preisreduzierung für Speichereinheiten ist möglich, wenn die richtige Speicherart ausgewählt wird. 
{\cite{AMZ08,AMZ09}} 
\\
Diese Arbeit legt den Fokus auf die Optimierung der oben genannten Dienste.
Als Überwachungswerkzeuge für die Kosten werden die AWS CloudWatch, der AWS Cost-Explorer und der AWS Trusted Advisor untersucht. 
\end{comment}
\subsection{Struktur der Arbeit}
%\addcontentsline{toc}{subsection}{Struktur der Arbeit}

%Schlüsselbegriffe
%Zunächst wird es eine kurze Einführung in die relevanten Begriffe geben, die für die zu untersuchenden AWS-Cloud-Services wichtig sind.

Diese Bachelorarbeit ist in folgende Kapitel unterteilt:\\\\
%2 Die gängigsten Cloud-Dienste, bei deren Geld verschwendet wird.
\textbf{Kapitel~\ref{kap_grundlagen}} 
befasst sich mit dem Begriff Cloud-Economy und erläutert das Potenzial der Cloud-Diensten im wirtschaftlichen Sinne. Die Cloud-Dienste EC2-Instanzen und S3 Speichereinheiten werden ebenfalls kurz erklärt. %Diese dienen als Grundlage für diese Arbeit. 
\\\\
%4 Überwachung von Kosten
\textbf{Kapitel~\ref{kap_zahlungsmodelle}} 
zeigt die verschiedenen Zahlungsmodelle für EC2-Instanzen. Es werden Kriterien vorgestellt, die helfen sollen, sich für das richtige Zahlungsmodell bei verschiedenen Szenarien zu entscheiden. 
\\\\
%5 AWS Cost Explorer und AWS-Kosten- und Nutzungsbericht
\textbf{In Kapitel~\ref{kap_kostenüberwachung }} werden die Werkzeuge eingeführt, die zur Überwachung der Kosten von Cloud-Diensten eingesetzt werden.
\\\\
%4 Methoden zur Kostenbremse
%4.1 S3 Intelligent-Tiering
%4.2 Instance-Scheduler für EC2 und AWS Reserved-Instance
\textbf{Kapitel~\ref{kap_Optimierung}} befasst sich mit Optimierungsmaßnahmen %insbesondere 
für EC2-Instanzen und S3 Speichereinheiten.

%Testumgebung
%Schließlich werden anhand eines Fallbeispiels in \textbf{Kapitel 5?}, die oben genannten Werkzeuge und Techniken in einer %kostenlosen Testumgebung getestet. Um die Zuverlässigkeit der Ergebnisse zu gewährleisten, wird alles gemacht, um die %Vorher- und Nachher-Szenarien vergleichbar zu machen. Dabei werden die Anzahl der Instanzen und deren Auslastung sowie %die Daten auf den Speichereinheiten berücksichtigt.
 
