Das Hauptziel dieses Kapitels besteht darin, Transparenz über die Kosten zu schaffen.

Es werden Dienste versucht zu identifizieren, die Optimierungspotenzial haben. 
\\
%Las razones pueden ser el simple hecho de haber olvidado apagar una instancia.
\subsection{Werkzeuge}
(To-Do: Intro)\\

Welche Metriken und Informationen lassen sich mit den hier erwähnten Werkzeuge finden?
Welche Kosten sind mit der Nutzung dieser Werkzeuge verbunden?


Es sollte gezeigt werden, die Arten/Kategorien von Kunden, die kostenlosen Zugang zu diesen Werkzeuge haben?

%WAS MIT AWS Budgets? AWS Bill and cost Man.?

\subsubsection{AWS CloudWatch}
%https://aws.amazon.com/de/cloudwatch/pricing/
%https://docs.aws.amazon.com/AmazonCloudWatch/latest/monitoring/GettingStarted.html

Amazon CloudWatch ermöglicht die Überwachung der Leistung von Resources, auch bei Ressourcen, die über verschiedene Regionen verteilt sind. 
CloudWatch sammelt operative Daten für die Verlaufsanalysen und die Entscheidungsfindung in Bezug auf Optimierung und Fehlerbehebung.

Eine der Metriken, die mit Amazon CloudWatch überwacht werden kann, ist die CPU-Last von EC2-Instanzen. 
Basierend auf einem Prozentsatz der CPU-Last können Alarme?Benachrichtigungen?[WELCHES WORT?] und Aktionen konfiguriert werden
\\\\
Zum Beispiel, eine dieser Aktionen ist die automatische Einrichtung neuer Instanzen zur Deckung des Kapazitätsbedarfs
\footnote{\cite{AWS1}, Seite 185}. 
Diese Art von Aktionen werden im Kapitel 7 Optimierungsmöglichkeiten tiefer behandelt.
\\\\
Im Folgenden werden die grundlegenden Bereiche und Begriffe von CloudWatch erläutert und wie sie zur Überwachung von Informationen über AWS-Ressourcen verwendet werden.
\\\\
\textbf{Visualisierung/ Dashboards}\\
Mit Cloud-Watch Dashboards können Boards[BESSERE FORMULIERUNG] mit relevanten Metriken angelegt werden. Sie ermöglichen eine umfassende Visualisierung von Metriken der Ressourcen in der AWS-Umgebung. Durch die Boards[PASSENDES WORT?], können auch Alarmen erstellt werden. Für die Einrichtung der Alarme ist kein technisches Wissen benötigt\footnote{\cite{AMZ14}, Seite 28}.

Die von CloudWatch gesammelte Informationen sind nicht nur relevant für System-Administratoren, es könnten auch für andere Personen innerhalb oder außerhalb der Organisation relevant sein.
Für sollte Fällen bietet CloudWatch die Möglichkeit, eine gemeinsame Sicht auf kritische Ressourcen- und Anwendungsmessungen zu generieren. 
Dies ermöglicht einen schnelleren Kommunikationsfluss in Echtzeit. Die Zugriffsverwaltung für geteilte Dashboards wird über AWS Identity and Access Management abgewickelt\footnote{\cite{AMZ14}, Seite 18 und 39}.
\\
%\begin{center}
  %    \includegraphics[scale=0.7]{sources/Name}%\label{fig:Name}\\
   %   \textbf{Abbildung \autoref{fig:Name}:} 
   %   Name
    %  %\footnote{Vgl. u.a.\cite{AMZ01}}
 % \end{center}
%\\\\

\textbf{Metriken} \\
Eine Metrik stellt eine Reihe von Daten über die Leistung einer Ressource in zeitlicher Reihenfolge dar. Standardmäßig werden viele kostenlose Metriken an CloudWatch übermittelt.
\\
Metriken bleiben wie folgt verfügbar:\\
Auflösung/Zeitraum: \\
1 Minute - 15 Tage\\
5 Minuten - 63 Tage.\\
1 Stunde - 15 Monate 
\footnote{\cite{AMZ15}, Wie lange werden die verfügbaren Metriken aufbewahrt?}

Ein Beispiel für eine Metrik ist die stündliche durchschnittliche Anzahl von Anfragen für eine bestimmte API, die wir in unserer Umgebung haben könnten.
Für eine detailliertere Überwachung ist es möglich, benutzerdefinierte Metriken zu konfigurieren, die eine Auflösung von bis zu 1 Sekunde zulassen. Ein praktisches Beispiel für benutzerdefinierte Metriken ist die Messung der Ladezeit einer Website.
[EIN BEISPIEL MIT BEZUG AUF K:OPTIMIERUNG?oder WIESO IST DIE AUFLÖSUNG RELEVANT?]

%\textbf{Logs} \\ BRAUCHE ICH DAS?
\textbf{Ereignisse / Events} \\
Bei CloudWatch ist ein Ereignis eine Änderung bei einer Ressource in der AWS-Umgebung. 
AWS-Ressourcen können Ereignisse erzeugen, wenn sich ihr Status ändert. 

Beispielsweise, ein Ereignis wird erzeugt, wenn Amazon EC2 Auto Scaling, Instanzen startet oder beendet \footnote{\cite{AMZ13}, Seite 1}. 

\textbf{Regel} \\
Eine Regel ordnet eintreffende Ereignisse zu und leitet diese zur Verarbeitung an Ziele weiter.
Eine einzelne Regel kann an mehrere Ziele weiterleiten, die alle parallel verarbeitet werden\footnote{\cite{AMZ13}, Seite 2}.

\textbf{Target / Ziele} \\
Ziele sind Ressourcen, die aufgerufen werden, wenn eine Regel ausgelöst wird.
EC2 instances, AWS Lambda functions und Amazon SNS topics sind unter anderem mögliche Ziele.
Die Ziele einer Regel müssen sich in derselben Region wie die Regel befinden
\footnote{\cite{AMZ13}, Seite 2}.
%Alle zugelassene Ressourcen \footnote\url{https://docs.aws.amazon.com/AmazonCloudWatch/latest/events/EventTypes.html}

\textbf{Alarme?Benachrichtigungen?}\\
%https://docs.aws.amazon.com/AmazonCloudWatch/latest/monitoring/AlarmThatSendsEmail.html
Benachrichtigt zu werden ist es wichtig, um relevante Ereignisse nicht zu verpassen und rechtzeitig Maßnahmen zu ergreifen. Mit CloudWatch können Alarme einrichtet werden, die durch Metriken wie die CPU-Last und auch Gebühren[ANDERES WORT] auf AWS-Rechnungen ausgelöst werden.

Benachrichtigungen können durch Amazon SNS oder zu einer E-Mail-Adresse geschickt werden.[NUR?]

%\\\\
\\\\
%This could be a good example %https://youtu.be/__knpcBRLHg?t=770
\\\\
%https://docs.aws.amazon.com/autoscaling/ec2/userguide/Cooldown.html
Autoscaling group seted with cooldown periods to avoid too much instances to by launched.

%Most popular parts of CloudWatch:
%https://youtu.be/k7wuIrHU4UY?t=785

% LINKS :

%https://docs.aws.amazon.com/AmazonCloudWatch/latest/monitoring/GettingStarted.html
%https://docs.aws.amazon.com/AmazonCloudWatch/latest/monitoring/WhatIsCloudWatch.html

%https://docs.aws.amazon.com/AmazonCloudWatch/latest/monitoring/Install-CloudWatch-Agent.html

%BEISPIEL VON BILLING ALARM
\subsubsection{Fakturierungsalarms mit CloudWatch}
%Seite 145 Amazon CloudWatch - Benutzerhandbuch
AWS CloudWatch empfängt Abrechnungsmetriken für alle Ressourcen. Auf der Grundlage dieser Metriken ist es daher möglich, Regeln zu erstellen, die bei Überschreitung des geplanten Budgets einen Alarm auslösen.
[KANN MAN NUR Total Estimated Charge verwenden ODER KÖNNTE MAN NACH PROJEKTE TRENNEN?]
[WIE KANN MAN EIN BEZUG AUF PROJEKTMANAGEMENT/Kostenkontrolle MACHEN?]

\subsubsection{Alarm bei Hoch- und Runterfahren von EC2-Instanzen}
Obwohl Auto-Scaling dafür sorgt, die Rechenkapazität dynamisch anzupassen, ist es von größter Wichtigkeit, über Änderungen in der Infrastruktur informiert zu sein, ohne die Dashboards manuell überprüfen zu müssen.
WEIL ....

\subsubsection{AWS Cost-Explorer}
...
%(How to use Spots and on-demand)
%detect CPU Utilization, with Amazon CloudWatch

\subsubsection{AWS Trusted advisor}
%https://aws.amazon.com/de/premiumsupport/technology/trusted-advisor/
%Kunden von AWS Basic Support und AWS Developer Support können auf grundlegende Sicherheitsprüfungen und alle Prüfungen für Servicekontingente zugreifen. Kunden von AWS Business Support und AWS Enterprise Support können auf alle Prüfungen zugreifen, einschließlich Kostenoptimierung, Sicherheit, Fehlertoleranz, Leistung und Servicekontingente. 
%was macht dieses?

%\begin{quote}
\textbf{Die 5 Kategorien von dem Trusted Advisor:}

\begin{itemize}
  \item
        Kostenoptimierung

  \item
        Leistung
  \item
        Sicherheit
  \item
        Fehlertoleranz
  \item
        Leistungsgrenzen.
\end{itemize}\textbf{}

%IST taging EINE STRATEGIE?
%\subsubsection{Für Speicher?}
%\subsubsection{Für VMs?}
%\subsubsection{Für DB?}

% \footnote{Vgl. u.a. \cite{RH1}}
%\end{quote}

%\begin{quote}
%  xxx
%\end{quote}\footnote{Vgl. u.a. \cite{AX1}}

%I could compute the cost of a query, user, transaction
% https://youtu.be/qYHR_V1lvNU?t=375