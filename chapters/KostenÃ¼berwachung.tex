Das Hauptziel dieses Kapitels besteht darin, Transparenz über die Kosten zu schaffen.

Es werden Dienste versucht zu identifizieren, die Optimierungspotenzial haben. 
\\
%Las razones pueden ser el simple hecho de haber olvidado apagar una instancia.
\subsection{Werkzeuge}
(To-Do: Intro)\\


Welche Metriken und Informationen lassen sich mit den hier erwähnten Werkzeuge finden?
Welche Kosten sind mit der Nutzung dieser Werkzeuge verbunden?


Es sollte gezeigt werden, die Arten/Kategorien von Kunden, die kostenlosen Zugang zu diesen Werkzeuge haben?

%WAS MIT AWS Budgets? AWS Bill and cost Man.?

\subsubsection{AWS CloudWatch}
%https://aws.amazon.com/de/cloudwatch/pricing/
%https://docs.aws.amazon.com/AmazonCloudWatch/latest/monitoring/GettingStarted.html

Amazon CloudWatch ermöglicht die Überwachung der Leistung von Reourcen, auch bei Ressourcen, die über verschiedene Regionen verteilt sind. 

%Es ist relevant für alle die Cloud-Dienste verbrauchen, wie DevOps-Manager, Ingenieure und Entwickler[IST DIESER SATZ RELEVANT ODER EINFACH FÜLLUNG?]. 

CloudWatch sammelt operative Daten für die Verlaufsanalysen und die Entscheidungsfindung.

Eine der Metriken, die mit Amazon CloudWatch überwacht werden kann, ist die CPU-Last von EC2-Instanzen. 
Basierend auf einem Prozentsatz der CPU-Last können Alarme?Benachrichtigungen?[WELCHES WORT?] und Aktionen konfiguriert werden
\\\\
Zum Beispiel die automatische Einrichtung neuer Instanzen zur Deckung des Kapazitätsbedarfs\footnote{\cite{AWS1}, Seite 185}. 
Diese Art von Aktionen werden im Kapitel 7 Optimierungsmöglichkeiten behandelt.
\\\\
\textbf{Visualisierung}\\
Mit CloudWatch-Dashboards können Metriken und historische Daten auf einem einzigen Dashboard nebeneinander grafisch dargestellt werden
{\cite{AWZ12}.

%\begin{center}
  %    \includegraphics[scale=0.7]{sources/Name}%\label{fig:Name}\\
   %   \textbf{Abbildung \autoref{fig:Name}:} 
   %   Name
    %  %\footnote{Vgl. u.a.\cite{AMZ01}}
 % \end{center}
%\\\\
\textbf{Alarme?Benachrichtigungen?}\\
%https://docs.aws.amazon.com/AmazonCloudWatch/latest/monitoring/AlarmThatSendsEmail.html
Mit CloudWatch können Alarme einrichtet werden, die durch Metriken wie die CPU-Last und auch Gebühren auf AWS-Rechnungen ausgelöst werden.

Benachrichtigt zu werden ist es wichtig, um relevante Ereignisse nicht zu verpassen und rechtzeitig Maßnahmen zu ergreifen.
\\\\
BEISPIEL VON BILLING ALARM
%https://docs.aws.amazon.com/autoscaling/ec2/userguide/Cooldown.html
Autoscaling group seted with cooldown periods to avoid too much instances to by launched.




Most popular parts of CloudWatch:
%https://youtu.be/k7wuIrHU4UY?t=785
\\
Dashboards:
\\
Alarms: notificar o actuar
\\
Metrics
\\
Logs
\\
Events: para on/off EC2
Lo que nos interesa es "Billing"


Para EC2, podemos ver la carga a CPU y luego escalar.
%https://docs.aws.amazon.com/AmazonCloudWatch/latest/monitoring/GettingStarted.html
%https://docs.aws.amazon.com/AmazonCloudWatch/latest/monitoring/WhatIsCloudWatch.html

%https://docs.aws.amazon.com/AmazonCloudWatch/latest/monitoring/Install-CloudWatch-Agent.html


\subsubsection{AWS Cost-Explorer}
...
%(How to use Spots and on-demand)
%detect CPU Utilization, with Amazon CloudWatch

\subsubsection{AWS Trusted advisor}
%https://aws.amazon.com/de/premiumsupport/technology/trusted-advisor/
%Kunden von AWS Basic Support und AWS Developer Support können auf grundlegende Sicherheitsprüfungen und alle Prüfungen für Servicekontingente zugreifen. Kunden von AWS Business Support und AWS Enterprise Support können auf alle Prüfungen zugreifen, einschließlich Kostenoptimierung, Sicherheit, Fehlertoleranz, Leistung und Servicekontingente. 
%was macht dieses?

%\begin{quote}
\textbf{Die 5 Kategorien von dem Trusted Advisor:}

\begin{itemize}
  \item
        Kostenoptimierung

  \item
        Leistung
  \item
        Sicherheit
  \item
        Fehlertoleranz
  \item
        Leistungsgrenzen.
\end{itemize}\textbf{}

%IST taging EINE STRATEGIE?
%\subsubsection{Für Speicher?}
%\subsubsection{Für VMs?}
%\subsubsection{Für DB?}

% \footnote{Vgl. u.a. \cite{RH1}}
%\end{quote}

%\begin{quote}
%  xxx
%\end{quote}\footnote{Vgl. u.a. \cite{AX1}}

%I could compute the cost of a query, user, transaction
% https://youtu.be/qYHR_V1lvNU?t=375