% Abstract (ACHTUNG: Abweichung zur Reihenfolge im Merkblatt!)
\begin{abstract}
    %\item Ziel der Arbeit

    Diese Arbeit ist für Nutzer von Cloud-Diensten relevant, die den Wechsel von klassischen Modellen bekannt als On-Premise zu On-Demand in der Cloud basierten Modelle planen und die unvorhersehbaren Kosten fürchten, die sich ihrer Kontrolle entziehen können. Es ist besonders interessant für Teams, die Cloud-Dienste in aktuellen Projekten verwalten und ihre Kosten optimieren wollen. Wenn man die Kosten für Cloud-Dienste wie alle anderen Kosten  betrachtet, ist es nur konsequent, über ihre Kontrolle und Optimierung nachzudenken.
    \\\\
    %\item Fragestellung der Arbeit
    In dieser Arbeit wird es gezeigt, wie mehr Kontrolle über die Kosten von Cloud-Diensten erhalten wird, indem sie überwacht werden.
    In Kombination damit werden Maßnahmen undWerkzeuge untersucht, die zu erheblichen Kosteneinsparungen in der Cloud führen.
    \\\\
    % \item Herangezogener, theoretischer Ansatz ("Quellen")
    Auf der Grundlage von Empfehlungen von Amazon Web Services selbst und aktuelle Fachliteratur.
    %  \item \textit{Optional:} Methodik

\end{abstract}

\renewcommand{\abstractname}{Abstract}
\begin{abstract}
    Platz für das englische Abstract...
\end{abstract}

