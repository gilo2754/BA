% Abstract (ACHTUNG: Abweichung zur Reihenfolge im Merkblatt!)
\begin{abstract}
[Rev ALL]In dieser Arbeit werden Werkzeuge und Maßnahmen untersucht, die die Kostenkontrolle über AWS-Dienste verbessern sollten. Darüber hinaus werden allgemeine Optimierungsmaßnahmen aufgezeigt, die bereits über die Jahre hinweg von anderen Nutzern getestet wurden und von Amazon Web Services (als Best Practices) empfohlen werden.
Die Grundlage dieser Untersuchung sind Empfehlungen von Cloud-Anbietern bezüglich Kostenüberwachung und -optimierung, Erfahrungen von Experten in dem Fachgebiet und aktuelle Fachliteratur.
\\\\
%fokusiert auf EC2+S3
Es ist besonders interessant für Teams, die AWS-Cloud-Dienste in aktuellen Projekten nutzen und ihre Kosten in der Cloud besser verstehen und optimieren wollen. Wenn die Kosten für Cloud-Dienste wie alle anderen Kosten betrachtet werden, ist es konsequent, über ihre Überwachung, Kontrolle und Optimierung nachzudenken. Ein häufiges Problem im Unternehmen ist, dass Entstehung von Kosten in der Cloud nicht verstanden werden. Dieses entzieht die Kontrolle über die Kosten von Cloud-Diensten???. Aus diesem Grund stehen Unternehmen, die bereits On-premise IT-Infrastruktur nutzen, einem Wechsel kritisch gegenüber, obwohl ihnen die Flexibilität von Cloud-Diensten bessere Wettbewerbsvorteile bieten würde. Deshalb sind die in dieser Arbeit aufgezeigten Werkzeuge und Maßnahmen relevant für diejenigen, die von einem Wechsel von klassischen Modellen, bekannt als On-Premise, zu cloudbasierten Modellen profitieren möchten.
%die zu erheblichen Kosteneinsparungen in der Cloud führen. 
% Angefangen bei der Wahl des richtigen Zahlungsmodells, über das automatische Herunterfahren ungenutzter Instanzen zu bestimmten Zeiten bis hin zur Implementierung von Autoscaling für EC2-Instanzen.
%WELCHE INFOS BRAUCHEN DIE CLOUD-NUTZER ÜBER DIE kOSTEN; UM DIE RICHTIGE ENTSCHIEDEN ZU TREFFEN.
\end{abstract}
\newpage
\renewcommand{\abstractname}{Abstract}
\begin{abstract}
    Platz für das englische Abstract....
\end{abstract}

