\begin{abstract}In dieser Arbeit werden Werkzeuge und Maßnahmen untersucht, die zur Kostenkontrolle von \textit{AWS-Diensten(Amazon Web Services)} beitragen. Darüber hinaus werden allgemeine Optimierungsmaßnahmen aufgezeigt, die bereits über die Jahre hinweg von anderen \textit{Cloud-Nutzern} getestet wurden. Die Optimierungsmaßnahmen werden von AWS als \textit{Best Practices} empfohlen. Die Grundlage dieser Arbeit sind Empfehlungen von Cloud-Anbietern bezüglich Kostenüberwachung und -optimierung, Erfahrungen von Experten dieses Fachgebiets und Anregungen aktueller Fachliteratur.\\\\Diese Arbeit ist besonders bedeutsam für Teams, die AWS-Cloud-Dienste in aktuellen Projekten anwenden und die Kosten in der Cloud besser verstehen und optimieren möchten. Wenn die Kosten für Cloud-Dienste wie alle anderen Kosten betrachtet werden, ist es konsequent, über ihre Überwachung, Kontrolle und Optimierung nachzudenken. Ein häufiges Problem in Unternehmen ist das fehlende Verständnis der in der Cloud anfallenden Kosten.\footnote{Vgl. Stern, Adam, 2018, The Truth About Cloud Pricing, o.S. \cite{SP1}}% Dieses entzieht die Kontrolle über die Kosten von Cloud-Diensten. 
%Entstehung von Kosten in der Cloud nicht verstanden werden. 
Aus diesem Grund stehen Unternehmen, die noch eine \textit{On-premise IT-Infrastruktur} nutzen, einem Wechsel kritisch gegenüber, obwohl ihnen die Flexibilität von Cloud-Diensten bessere Wettbewerbsvorteile bieten würde. 
%Gerade für Unternehmen, die den Nutzen/Vorteile solcher Dienste erkennen und in Erwägung ziehen von den klassischen (bekannt als ...) zu cloudbasierten Modellen zu wechseln, erscheint diese Arbeit aufgrund der aufgezeigten Werkzeuge und Maßnahmen relevant. 
Deshalb sind die in dieser Arbeit aufgezeigten Werkzeuge und Maßnahmen für Unternehmen relevant, die von einem Wechsel von klassischen Modellen (bekannt als On-Premise) zu cloudbasierten Modellen profitieren möchten.
%die zu erheblichen Kosteneinsparungen in der Cloud führen. 
% Angefangen bei der Wahl des richtigen Zahlungsmodells, über das automatische Herunterfahren ungenutzter Instanzen zu bestimmten Zeiten bis hin zur Implementierung von Autoscaling für EC2-Instanzen.
%WELCHE INFOS BRAUCHEN DIE CLOUD-NUTZER ÜBER DIE kOSTEN; UM DIE RICHTIGE ENTSCHIEDEN ZU TREFFEN.
\end{abstract}
%\newpage
%\renewcommand{\abstractname}{Abstract}
%\begin{abstract}
 %   Platz für das englische Abstract....
%\end{abstract}

