% Abstract (ACHTUNG: Abweichung zur Reihenfolge im Merkblatt!)
\begin{abstract}
In dieser Arbeit werden Werkzeuge untersucht, die einen klareren Überblick über die finanziellen %und leistungsbezogenen
Ressourcen schaffen. Mit den gesammelten Informationen dienen sie dazu, direkte Maßnahmen zu ergreifen. Darüber hinaus werden allgemeine Optimierungsmaßnahmen aufgezeigt, die bereits über die Jahre hinweg von anderen Nutzern getestet wurden und von Amazon Web Services (als Best Practices) empfohlen werden.
%die zu erheblichen Kosteneinsparungen in der Cloud führen. 
% Angefangen bei der Wahl des richtigen Zahlungsmodells, über das automatische Herunterfahren ungenutzter Instanzen zu bestimmten Zeiten bis hin zur Implementierung von Autoscaling für EC2-Instanzen.
%WELCHE INFOS BRAUCHEN DIE CLOUD-NUTZER ÜBER DIE kOSTEN; UM DIE RICHTIGE ENTSCHIEDEN ZU TREFFEN.
Die Grundlage dieser Recherche sind Empfehlungen von Cloud-Anbietern bezüglich Kostenüberwachung und -optimierung, Erfahrungen von Experten in dem Fachgebiet und aktuelle Fachliteratur.
\\\\
Es ist besonders interessant für Teams, die Cloud-Dienste in aktuellen Projekten nutzen und ihre Kosten in der Cloud besser verstehen und optimieren wollen. Wenn die Kosten für Cloud-Dienste wie alle anderen Kosten betrachtet werden, ist es konsequent, über ihre Kontrolle und Optimierung nachzudenken. Ein häufiges Problem ist, dass Kosten entstehen, die sich der Kontrolle der Nutzer entziehen.%[ZITAT?]
Aus diesem Grund stehen Unternehmen die bereits On-Premise IT-Infrastruktur nutzen, einem Wechsel kritisch gegenüber, obwohl ihnen die Flexibilität von Cloud-Diensten bessere Wettbewerbsvorteile bieten würde.
Deshalb sind die in dieser Arbeit aufgezeigten Werkzeuge und Maßnahmen relevant für diejenigen, die von einem  Wechsel von klassischen Modellen, bekannt als On-Premise, zu Cloud basierten Modellen profitieren möchten.
\end{abstract}
\newpage
\renewcommand{\abstractname}{Abstract}
\begin{abstract}
    Platz für das englische Abstract....
\end{abstract}

