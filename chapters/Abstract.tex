% Abstract (ACHTUNG: Abweichung zur Reihenfolge im Merkblatt!)
\begin{abstract}
    %\item Ziel der Arbeit
    Diese Arbeit beschäftigt sich damit, wie mehr Kontrolle über die Kosten von Cloud-Diensten erhalten wird, indem sie überwacht werden.
    In Kombination damit werden Maßnahmen und Werkzeuge untersucht, die zu erheblichen Kosteneinsparungen in der Cloud führen. 
    \\
    Angefangen bei der Wahl des richtigen Zahlungsmodells, über das automatische Herunterfahren ungenutzter Instanzen zu bestimmten Zeiten bis hin zur Implementierung von Autoscaling .
    %WELCHE INFOS BRAUCHEN DIE CLOUD-NUTZER ÜBER DIE kOSTEN; UM DIE RICHTIGE ENTSCHIEDEN ZU TREFFEN.

    % \item Herangezogener, theoretischer Ansatz ("Quellen")
    Die Arbeit ist auf der Grundlage von Empfehlungen von Amazon Web Services selbst, Erfahrungen von Experten in dem Fachgebiet und aktuelle Fachliteratur geschrieben.
    \\\\
    Diese Arbeit ist für Nutzer von Cloud-Diensten relevant, die den Wechsel von klassischen Modellen bekannt als On-Premise zu On-Demand in der Cloud basierten Modelle planen und die unvorhersehbaren Kosten fürchten, die sich ihrer Kontrolle entziehen können. Es ist besonders interessant für Teams, die Cloud-Dienste in aktuellen Projekten verwalten und ihre Kosten optimieren wollen. Wenn die Kosten für Cloud-Dienste wie alle anderen Kosten betrachtet werden, ist es nur konsequent, über ihre Kontrolle und Optimierung nachzudenken.
\end{abstract}

\newpage
\renewcommand{\abstractname}{Abstract}
\begin{abstract}
    Platz für das englische Abstract...
\end{abstract}

