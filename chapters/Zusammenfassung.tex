(To-Do:)
\\
\subsection{Bewusstsein in der gesamten Organisation}
Zusätzlich zu den bisher genannten Maßnahmen ist es wichtig, dass Verantwortliche für die Kostenerzeugung Bewusstsein entwikclen. Von dem Entwickler bis zum der IT-Manager, jeder sollte wissen, dass es so einfach ist, Cloud-Dienste mit ein paar Klicks zu beauftragen. Diese können in kurzer Zeit unglückliche/ungeplante Kosten verursachen. 
\\\\
\subsection{Die richtige Personen(Owneship verbreiten)}
Die technischen Maßnahmen zur Überwachung und Kostenreduzierung wurden dargelegt, aber jemand muss diese Analysen, Anpassungen und Entscheidungen durchführen. 
Deshalb ist es wichtig, bestimmte Personen zu berücksichtigen, die die Verantwortung für das Geschehen in den Cloud-Systemen übernehmen. Idealerweise Menschen, die sich für das Thema interessieren und über die notwendigen Kenntnisse verfügen, um die gesetzten Ziele zu erreichen. 
%https://content.aws.training/video/cmcfrm/de/x2/1.0.0/jwplayer.html?endpoint=https%3a%2f%2flrs.aws.training%2fTCAPI%2f&auth=Basic%20OjUzYmEwYTZmLTk0ZmMtNDAwZi1hODBlLWQ1YzA5NmNkOWY1MA%3d%3d&actor=%7b%22objectType%22%3a%22Agent%22%2c%22name%22%3a%5b%22zEjHPzGX10miDWp26Y_cLg2%22%5d%2c%22mbox%22%3a%5b%22mailto%3alms-user-zEjHPzGX10miDWp26Y_cLg2%40amazon.com%22%5d%7d&registration=2f22bc75-44b9-4175-a976-5ba4d7fe2902&activity_id=http%3a%2f%2fid.tincanapi.com%2factivity%2ftincan-prototypes%2fgolf-example&grouping=http%3a%2f%2fid.tincanapi.com%2factivity%2ftincan-prototypes%2fgolf-example&content_token=2554ffa1-5a1e-4737-9a0f-fc3abda1083e&content_endpoint=https%3a%2f%2flrs.aws.training%2fTCAPI%2fcontent%2f&externalRegistration=CompletionThresholdPercent%7c80!InstanceId%7c0!PackageId%7ccmcfrm_de_x2_1.0.0!RegistrationTimestampTicks%7c16324112989178350!SaveCompletion%7c1!TranscriptId%7cl4fipkeHAkKh-aGlnjYdug2!UserId%7czEjHPzGX10miDWp26Y_cLg2&externalConfiguration=&width=1366&height=728&left=0&top=0
%2:25
\begin{comment}

\newpage
\vspace{1cm}
\begin{tcolorbox}[title={Inhalte der \textit{Zusammenfassung und Ausblick}}]
  Das Kapitel \textit{Zusammenfassung und Ausblick} enthält folgende formale Aspekte\footnote{Vgl. \cite{BBoJ},S. 6}:
  \begin{itemize}
    \item Kapitelweise Kurzdarstellung der Inhalte (inklusive Referenzierung auf die Kapitelnummerierung) => Nach dem Motto: \textit{Was wurde wo beschrieben?}
    \item Kurzdarstellung \textit{Problem – Lösungsweg – Ergebnisse}
    \item Rückkopplung auf die Einleitung: Wurde die Zielstellung der Arbeit und die Fragestellung zufriedenstellend beantwortet?
    \item Kritische Bewertung (sofern nicht bereits im Hauptteil geschehen)
    \item Offene Probleme
    \item Richtung der zukünftigen/möglichen Arbeiten
    \item Erläuterung, warum welche Aspekte in der Arbeit nicht erläutert wurden
  \end{itemize}
Von Buch "Gestaltung"
  Schluss (Fazit)
Den Abschluss der Arbeit bildet die Zusammenfassung der wesentlichen
Ergebnisse, die folgende drei Punkte beinhaltet:
Beantwortung der Forschungsfrage, die Sie in der Einleitung
aufgeworfen haben.
Sinnstiftung der Arbeit: Für welchen Zweck sollen die Ergebnisse
verwendet werden?
Gegebenenfalls auch persönliche Bemerkungen und Bewertungen oder
ein kurzer Ausblick.
\end{tcolorbox}

\end{comment}

%FAZIT ist (Von Schribe)
%Was ist? Hier sollten die neue Erkenntnisse der Arbeit dargestellt wrden

%TIPPS: 
%Vermeide "man" und "ich"
%Addressanten sind potenzielle Cloud-Nutzer/IT-Personal.
%Es wurde AWS ausgewählt, weil...
%Wirtschaftliche Betrachtung, weil das wichtig für Firmen ist

%STRUKTUR:
%Zusammenfassung, von was gemacht wurde?
%Beanwortung der FF 
%Mehrwer für die Praxis
%Limitationen: es konnte in dieser Arbeit nicht in der Praxis geprüft werden, ob die Maßnahmen ihre Verprechen anhalten
%Weitere Forschungen: es empfehlt sich diese Maßnahmen in echte Systeme einzusetzen
