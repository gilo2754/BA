TEXT MUSS KONTROLIERT WERDEN...\\

\newpage
\vspace{1cm}
\begin{tcolorbox}[title={Inhalte der \textit{Zusammenfassung und Ausblick}}]
  Das Kapitel \textit{Zusammenfassung und Ausblick} enthält folgende formale Aspekte\footnote{Vgl. \cite{BBoJ},S. 6}:
  \begin{itemize}
    \item Kapitelweise Kurzdarstellung der Inhalte (inklusive Referenzierung auf die Kapitelnummerierung) => Nach dem Motto: \textit{Was wurde wo beschrieben?}
    \item Kurzdarstellung \textit{Problem – Lösungsweg – Ergebnisse}
    \item Rückkopplung auf die Einleitung: Wurde die Zielstellung der Arbeit und die Fragestellung zufriedenstellend beantwortet?
    \item Kritische Bewertung (sofern nicht bereits im Hauptteil geschehen)
    \item Offene Probleme
    \item Richtung der zukünftigen/möglichen Arbeiten
    \item Erläuterung, warum welche Aspekte in der Arbeit nicht erläutert wurden
  \end{itemize}
Von Buch "Gestaltung"
  Schluss (Fazit)
Den Abschluss der Arbeit bildet die Zusammenfassung der wesentlichen
Ergebnisse, die folgende drei Punkte beinhaltet:
Beantwortung der Forschungsfrage, die Sie in der Einleitung
aufgeworfen haben.
Sinnstiftung der Arbeit: Für welchen Zweck sollen die Ergebnisse
verwendet werden?
Gegebenenfalls auch persönliche Bemerkungen und Bewertungen oder
ein kurzer Ausblick.
\end{tcolorbox}

%FAZIT ist (Von Schribe)
%Was ist? Hier sollten die neue Erkenntnisse der Arbeit dargestellt wrden

%TIPPS: 
%Vermeide "man" und "ich"
%Addressanten sind potenzielle Cloud-Nutzer/IT-Personal.
%Es wurde AWS ausgewählt, weil...
%Wirtschaftliche Betrachtung, weil das wichtig für Firmen ist

%STRUKTUR:
%Zusammenfassung, von was gemacht wurde?
%Beanwortung der FF 
%Mehrwer für die Praxis
%Limitationen: es konnte in dieser Arbeit nicht in der Praxis geprüft werden, ob die Maßnahmen ihre Verprechen anhalten
%Weitere Forschungen: es empfehlt sich diese Maßnahmen in echte Systeme einzusetzen
