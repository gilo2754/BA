(To-Do:)
\\Kapitelweise Kurzdarstellung der Inhalte (inklusive Referenzierung auf die \\Kapitelnummerierung) => Nach dem Motto: \textit{Was wurde wo beschrieben?}
\\Kurzdarstellung \textit{Problem – Lösungsweg – Ergebnisse}
\\Rückkopplung auf die Einleitung: Wurde die Zielstellung der Arbeit und die \\Fragestellung zufriedenstellend beantwortet?
\\Kritische Bewertung (sofern nicht bereits im Hauptteil geschehen)
\\Offene Probleme/Themen: CloudFormation, IAM, Mit Serverless Kosten nach Transaktionen?,  AWS Cost Anomalies
\\Richtung der zukünftigen/möglichen Arbeiten
\\Erläuterung, warum welche Aspekte in der Arbeit nicht erläutert 

\subsection*{Bewusstsein in der gesamten Organisation entwickeln}
\addcontentsline{toc}{subsection}{Bewusstsein in der gesamten Organisation} % Manuellen Eintrag im Inhaltsverzeichnis erzeugen
Zusätzlich zu den bisher genannten Maßnahmen ist es wichtig, dass Verbraucher von Cloud-Diensten Bewusstsein für die Entstehung von Kosten entwikclen[ODER sensibilisier werden?]. Von dem Entwickler bis zum der IT-Manager, jeder sollte wissen, dass es so einfach ist, Cloud-Dienste mit ein paar Klicks zu beauftragen\footnote{Plusserver: Kostenoptimierung in AWS, S.5\cite{PS1}.}. Diese können in kurzer Zeit ungewünschte  Kosten verursachen oder sogar über Jahre hinweg wirtschaftliche Schäden verursachen. 
%Wer wird benachrichtigt. Welche Tools? https://docs.aws.amazon.com/de_de/AWSEC2/latest/UserGuide/monitoring_ec2.html
\\
\subsection*{Die richtige Personen finden, Ownership/Commitment verbreiten}
\addcontentsline{toc}{subsection}{Die richtige Personen finden, Owneship verbreiten} % Manuellen Eintrag im Inhaltsverzeichnis erzeugen
Die technischen Maßnahmen zur Überwachung und Kostenreduzierung wurden dargelegt, aber jemand muss diese Analysen, Anpassungen und Entscheidungen durchführen. 
Deshalb ist es wichtig, bestimmte Personen zu berücksichtigen, die die Verantwortung für das Geschehen in den Cloud-Systemen übernehmen. Idealerweise Menschen, die sich für das Thema interessieren und über die notwendigen Kenntnisse verfügen, um die gesetzten Ziele zu erreichen. 
%Sie redete über DIESES t.ly/XJ24
\\
\subsection*{5G/IoT generierte Daten}
\addcontentsline{toc}{subsection}{5G is comming} % Manuellen Eintrag im Inhaltsverzeichnis erzeugen
Mit 5G ist pronostiziert, dass mehr Daten[WIE VIELE / WANN?] automatisch und schnell von Maschinen produziert werden.
\\
\subsection*{Rentabilität bei der Optimierungsmaßnahmen}
\addcontentsline{toc}{subsection}{Rentabilität bei der Optimierungsmaßnahmen} % Manuellen Eintrag im Inhaltsverzeichnis erzeugen
Kostenoptimierung UND -Überwachung SOLLEN DIE Einsparungen NICHT ÜBERSCHREITEN . 
TRUSTED ADVISOR NICHT FÜR JEDE FIRMA.
%https://content.aws.training/video/cmcfrm/de/x2/1.0.0/jwplayer.html?endpoint=https%3a%2f%2flrs.aws.training%2fTCAPI%2f&auth=Basic%20OjUzYmEwYTZmLTk0ZmMtNDAwZi1hODBlLWQ1YzA5NmNkOWY1MA%3d%3d&actor=%7b%22objectType%22%3a%22Agent%22%2c%22name%22%3a%5b%22zEjHPzGX10miDWp26Y_cLg2%22%5d%2c%22mbox%22%3a%5b%22mailto%3alms-user-zEjHPzGX10miDWp26Y_cLg2%40amazon.com%22%5d%7d&registration=2f22bc75-44b9-4175-a976-5ba4d7fe2902&activity_id=http%3a%2f%2fid.tincanapi.com%2factivity%2ftincan-prototypes%2fgolf-example&grouping=http%3a%2f%2fid.tincanapi.com%2factivity%2ftincan-prototypes%2fgolf-example&content_token=2554ffa1-5a1e-4737-9a0f-fc3abda1083e&content_endpoint=https%3a%2f%2flrs.aws.training%2fTCAPI%2fcontent%2f&externalRegistration=CompletionThresholdPercent%7c80!InstanceId%7c0!PackageId%7ccmcfrm_de_x2_1.0.0!RegistrationTimestampTicks%7c16324112989178350!SaveCompletion%7c1!TranscriptId%7cl4fipkeHAkKh-aGlnjYdug2!UserId%7czEjHPzGX10miDWp26Y_cLg2&externalConfiguration=&width=1366&height=728&left=0&top=0
%2:25

\subsection*{Handlungsempfehlungen}
[SIND SIE HIER RICHTIG PLAZIERT? SOLLTEN LIEBER IN FAZIT SEIN?;NOCH ZU VERVOLSTÄNDIGEN]
\\\\
\subsubsection*{Handlungsempfehlung 1:} 
Es kann in Erwägung gezogen werden, für einen begrenzten Zeitraum von 3 Monaten einen Support-Plan zu bezahlen, um aus den gegebenen Empfehlungen zu lernen. Oder Business-Plan alle 6 Monate für 1 Monat zu aktivieren.  
\\\\
\subsubsection*{Handlungsempfehlung 2:} 
Ein Berater für eine Prüfung und Optimierung der AWS-Diensten kann in Deutschland zwischen x und N-EUR kosten. Dies ist eine Alternative zu den Plänen des Trusted-Advisor. Ein Berater, der alle 5 Kategorien abdeckt, könnte [BETRAG] kosten. %Das sagt der Berater Juanito von der Firma XXX.
\\\\
\begin{comment}
  \subsection*{Umweltbezogene Aspekte}
  \addcontentsline{toc}{subsection}{Umweltbezogene Aspekte} % Manuellen Eintrag im Inhaltsverzeichnis erzeugen
  [WORK IN PROGRESS]\\
  In dieser Arbeit geht es um die Überwachung und Optimierung von Cloud-Diensten unter finanziellen Aspekten. 
  
  Serverfarmen und ihre Speichereinheiten  haben Auswirkungen auf die Umwelt, da sie eine große Menge an Strom benötigen. 
  [t.ly/XYMJ]
  \\
  \subsection*{Test von den Werkzeugen und Maßnahmen}
  \addcontentsline{toc}{subsection}{Test von den Werkzeugen und Maßnahmen} % Manuellen Eintrag im Inhaltsverzeichnis erzeugen
  Da es in dieser Arbeit zeitlich nicht gelungen ist, die Überwachungswerkzeuge und Optimierungsmaßnahmen umzusetzen, bleibt es noch sie in einer echten Umgebung zu testen. Es wäre möglich zu verifizieren, ob die hier genannten Maßnahmen zur vergleichbaren Einsparungen führen, wie die vom Cloud-Anbieter Amazon genannten.
  
  Amazon bietet ein kostenloses Kontingent an, die jedoch für diese Tests nicht genug war. 
  \\
  \end{comment}
  
\begin{comment}
Von Buch "Gestaltung"
  Schluss (Fazit)
Den Abschluss der Arbeit bildet die Zusammenfassung der wesentlichen
Ergebnisse, die folgende drei Punkte beinhaltet:
Beantwortung der Forschungsfrage, die Sie in der Einleitung
aufgeworfen haben.
Sinnstiftung der Arbeit: Für welchen Zweck sollen die Ergebnisse
verwendet werden?
Gegebenenfalls auch persönliche Bemerkungen und Bewertungen oder
ein kurzer Ausblick.
\end{tcolorbox}

\end{comment}

%FAZIT ist (Von Schribe)
%Was ist? Hier sollten die neue Erkenntnisse der Arbeit dargestellt wrden

%TIPPS: 
%Vermeide "man" und "ich"
%Addressanten sind potenzielle Cloud-Nutzer/IT-Personal.
%Es wurde AWS ausgewählt, weil...
%Wirtschaftliche Betrachtung, weil das wichtig für Firmen ist

%STRUKTUR:
%Zusammenfassung, von was gemacht wurde?
%Beanwortung der FF 
%Mehrwer für die Praxis
%Limitationen: es konnte in dieser Arbeit nicht in der Praxis geprüft werden, ob die Maßnahmen ihre Verprechen anhalten
%Weitere Forschungen: es empfehlt sich diese Maßnahmen in echte Systeme einzusetzen
