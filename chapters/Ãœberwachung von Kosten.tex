\subsection{GraphQL}
\paragraph{}
GraphQL ist eine Abfragesprache und Server-Laufzeitumgebung für APIs.
Ihre Aufgabe ist es, genau die Daten zu liefern, die anfordert werden, und nicht mehr.
\\
Mit GraphQL sind APIs schnell, flexibel und einfach für Entwickler.
\\ \\
Laut dem 2020 State of the API Report von Postman.com steht GrapQL an fünfter Stelle der spannendsten Technologien für 2021.
  {Vgl. u.a. \cite{PM1}}

Im Hinblick auf die Art und Weise, wie Abfragen an den Server mithilfe von
\\ GraphQL behandelt werden können, sind folgende Aspekte zu beachten.
\\
\begin{quote}
  \textbf{Nachteile}
  \begin{itemize}
    \item
          Für Entwickler, die sich bereits mit REST-APIs auskennen, bedeutet GraphQL weiteren Lernaufwand.
    \item
          Mit GraphQL verschiebt sich die Funktionalität von Datenabfragen zur Serverseite, was zusätzliche Komplexität für Serverentwickler bedeutet.

  \end{itemize}

  \footnote{Vgl. u.a. \cite{RH1}}
\end{quote}

\subsubsection{GraphQL Playground}
Mit GraphQL Playground haben wir die Möglichkeit, alle Abfragen und Mutationen zu testen. Wir erhalten Zugriff auf relevante Informationen wie verfügbare Felder und deren Datentyp. Diese Informationen werden aktualisiert, wenn der Servercode geändert wurde. Dadurch wurde eine aktuelle API-Dokumentation gewährleistet. Für unser Projekt war es sehr praktisch und hat die Kommunikation als Entwickler effizienter gemacht.
\\
%\begin{center}
%\includegraphics[scale=0.60]{GraphQL_Playground}\label{fig:GraphQL_Playground}
%\end{center}
%\textbf{Abbildung \autoref{fig:GraphQL_Playground}:}
Genau derselbe Code wird für die Abfrage von Benutzerinformationen später verwendet.
\newpage


\textbf{Apollo Client}\\
Warum haben wir uns für Apollo entschieden? Was ist ApolloClient?
TODO
%Coming...
\newpage

\subsubsection{Leseabfrage}
Nachdem eine Abfrage exportiert wurde, ist sie bereit, in einer React-Komponente \\
importiert und angewendet zu werden.


\newpage

\subsubsection{Mutationen}
TODO

\newpage
\subsection{Axios}
Zusätzlich zu den GraphQL-Abfragen wurde eine Post-Anfrage mit Axios bereitgestellt.
Mit dieser war es möglich, Bilder auf die S3 Speicher von AWS hochzuladen.

\begin{quote}
  xxx
\end{quote}\footnote{Vgl. u.a. \cite{AX1}}
